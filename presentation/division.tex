\section{Division}
\begin{frame}[fragile]
  \frametitle{Division}
  \framesubtitle{Algorithm 1/3}
  The intuition behind the division algorithm:
        \begin{itemize}\footnotesize
            \item Multiply the dividend with the of inverse divisor.
                \pause
              \item Use shifts to represent the inverse as a big integer.
                \pause
              \item Approximate the shifted inverse by Newton's Method using \cite{watt2023efficient}.
                \pause
              \item Compute quotient and remainder to adjust approximation.
              \end{itemize}\vspace*{0.6em}\pause

              \begin{block}{Quotient of big integers by shifted inverse}\scriptsize
              We define the quotient of big integers $u\leq B^{h\in \mathbb{N}}$ and $v$ in base $B$ using \cite{watt2023efficient} as:\vspace*{-0.3em}
              \begin{equation}
    \scriptsize
    u~\mathtt{quo}~v = \mathtt{shift}_{-h}~ (u \cdot \mathtt{shinv}_h~v) + \delta,\quad \textbf{where~} \delta \in \{0,1\}
  \end{equation}\vspace*{-1.5em}
              \begin{equation}\scriptsize
    \mathtt{shift}_{n\in\mathbb{Z}}~u = \lfloor u \cdot B^n \rfloor \qquad \qquad \quad ~\,\, \mathtt{shinv}_{n\in\mathbb{N}}~v = \lfloor B^{n}/ v \rfloor
  \end{equation}
  \end{block}
      \end{frame}

      
\begin{frame}[fragile]
  \frametitle{Division}
  \framesubtitle{Algorithm 2/3}
  The algorithm of \cite{watt2023efficient} is based on the Netwon iteration:\vspace*{-0.5em}
  \begin{equation}\scriptsize
  x_{i+1} = x_i - \dfrac{f(x_i)}{f'(x_i)} = x_i + \left( x_i - \dfrac{v}{u}x^2_i \right),\quad \textbf{where}~x\in \mathbb{R}~ \textbf{and}~ f(x) = \dfrac{u}{x} - v
\end{equation}\pause
It is modified in \cite{watt2023efficient} w.r.t.\ three aspects:
\begin{itemize}\footnotesize
\item It is discretized to integers, so we compute $w\in\mathbb{Z}$ rather than $x\in\mathbb{R}$.\pause
\item $u$ is specialized to $B^h$ (where $B$ is the base and $u \leq B^h$).\pause
\item Use a shift rather than a division since $v/(B^h)=v\cdot B^{-h}\geq \mathtt{shift}_{-h}~v$.
\end{itemize}\pause
The Newton iteration becomes:\vspace*{-0.5em}
\begin{equation}\scriptsize
  w_{i+1} =  w_i + \mathit{shift}_{-h}\left( \mathit{shift}_{h}~w_i - v\cdot w^2_i \right) =  w_i + w_i\lfloor B^h - v\cdot w_i \rfloor B^{-h}
\end{equation}
\end{frame}

\begin{frame}[fragile]
  \frametitle{Division}
  \framesubtitle{Algorithm 3/3}

  Instead of showing the algorithm for computing the division and the shifted
  inverse, let us look at the Futhark implementation.\vspace*{0.75em}\pause

  In the thesis, the Futhark implementation was {\red invalid}.

  Now it {\green validates}!

  (Without the divisor prefixes and shorter iterates optimizations.)\vspace*{0.75em}\pause

  Hence all arithmetics are in full length.

  The analysis in \cite{watt2023efficient} gives work $O(\log (h-k)(M(h)+M(|h/2 - k|)))$.

  If we assume $h=m+k$, we get $O(\log (m)M(m))$.
\end{frame}


\begin{frame}[fragile]
  \frametitle{Division}
  \framesubtitle{Implementation 1/5}
  Revisions to section 9.2 and section 9.3 of the thesis:
  \begin{itemize}
  \item $2$ guard digits are sufficient.\pause
  \item The thesis states that $v > B^h$ corresponds to:
    \begin{equation}\footnotesize
    \exists i\in \mathbb{N}.~(h<i<m \land v[i] \neq 0) \lor (h = i<m \land v[i] > 1)
  \end{equation}
  This is not true. E.g.\ it fails on $\arr{1,0,0,1} > B^3$.\pause

  Instead, define $v > B^h$ as {\footnotesize $\neg(v < B^h \lor v = B^h)$}.

  Define $v < B^h$ as {\footnotesize $\forall i\in \mathbb{N}.~i \geq h \lor v[i] = 0$}.

  Define $v = B^h$ as {\footnotesize $\forall i\in \mathbb{N}.~(i=h \land v[i] = 1) \lor (i \neq h \land v[i] = 0)$}.
    \end{itemize}
  \end{frame}

\begin{frame}[fragile]
  \frametitle{Division}
  \framesubtitle{Implementation 2/5}
\begin{lstlisting}[language=futhark,basicstyle=\scriptsize,escapeinside=!!,frame=single]
def div [m] (u: [m]ui) (v: [m]ui) : ([m]ui, [m]ui) =
  let h = findh u !\green \texttt{-}\texttt{-} $u\leq B^{h}$!
  let k = findk v !\green \texttt{-}\texttt{-} $B^k \leq v < B^{k+1}$!

  let p = 2*(m + (i64.bool (k <= 1)) + (i64.bool (k == 0)))
  let up = map (\ i -> if i < m then u[i] else 0 ) (iota p)
  let vp = map (\ i -> if i < m then v[i] else 0 ) (iota p)

  let (h, k, up, vp) =
         if k == 1 then (h+1, k+1, shift 1 up, shift 1 vp)
    else if k == 0 then (h+2, k+2, shift 2 up, shift 2 vp)
    else                (h,   k,   up,         vp)

  let q = shinv k vp h |> mul up |> shift (-h) |> take m
  let r = mul v q |> sub u |> fst
  in if not (lt r v)
     then (add q (singleton m 1), fst (sub r v)) else (q, r)
\end{lstlisting}
\end{frame}

\begin{frame}[fragile]
  \frametitle{Division}
  \framesubtitle{Implementation 3/5}
\begin{lstlisting}[language=futhark,basicstyle=\scriptsize,escapeinside=@@,firstnumber=18,frame=single]
def shinv [m] (k: i64) (v: [m]ui) (h: i64) : [m]ui =
  assert (k > 1) (
       if gtBpow v h          then new m
  else if gtBpow (muld v 2) h then singleton m 1
  else if eqBpow v k          then bpow m (h - k)
  else
    let V = (toQi v[k-2])
            + (toQi v[k-1] << (i64ToQi bits))
            + (toQi v[k]   << (i64ToQi (2*bits)))
    let W = ((0 - V) / V) + 1
    let w = map (\i -> if i <= 1
                       then fromQi (W >> (i64ToQi (bits*i)))
                       else 0) (iota m)

    in if h - k <= 2 then shift (h - k - 2) w
                     else refine v w h k     )
\end{lstlisting}
\end{frame}

\begin{frame}[fragile]
  \frametitle{Division}
  \framesubtitle{Implementation 4/5}
\begin{lstlisting}[language=futhark,basicstyle=\scriptsize,escapeinside=@@,firstnumber=34,frame=single]
def refine [m] (v:[m]ui) (w:[m]ui) (h:i64) (k:i64) : [m]ui =
  let g = 1
  let h = h + g
  let (w, _) =
    loop (w, l) = (shift (h-k-2) w, 2) while h - k > l do
    let w = step h v w 0 l 0
    let l = i64.min (2*l-1) (h-k)
    in (w, l)
  in shift (-g) w
\end{lstlisting}\vspace*{0.5em}

\begin{lstlisting}[language=futhark,basicstyle=\scriptsize,escapeinside=@@,firstnumber=43,frame=single]
def step [m] (h: i64) (v: [m]ui) (w: [m]ui)
             (n: i64) (l: i64)   (g:i64)   : [m]ui =
  let (pwd, sign) = powdiff v w (h-n) (l-g)
  let wpwdS = shift (2*n - h) (mul w pwd)
  let wS = shift n w
  in if sign then fst (sub wS wpwdS) else add wS wpwdS
\end{lstlisting}
\end{frame}

\begin{frame}[fragile]
  \frametitle{Division}
  \framesubtitle{Implementation 5/5}
\begin{lstlisting}[language=futhark,basicstyle=\scriptsize,escapeinside=@@,firstnumber=49,frame=single]
def powdiff [m] (v: [m]ui) (w: [m]ui)
                (h: i64)   (l: i64)  : ([m]ui, bool) =
  let L = (prec v) + (prec w) - l + 1
  in if (ez v) || (ez w) then (bpow m h, false)
     else if L >= h then sub (bpow m h) (mul v w)
     else let P = multmod v w L
          in if ez P then (P, false)
             else if P[L-1] == 0 then (P, true)
             else sub (bpow m L) P
\end{lstlisting}\vspace*{0.5em}

\begin{lstlisting}[language=futhark,basicstyle=\scriptsize,escapeinside=@@,firstnumber=58,frame=single]
def multmod [m] (v: [m]ui) (w: [m]ui) (e: i64) : [m]ui =
  let vw = mul (take e v) (take e w)
  in map (\ i -> if i < e then vw[i] else 0 ) (iota m)
\end{lstlisting}
\end{frame}

\begin{frame}[fragile]
  \frametitle{Division}
  \framesubtitle{Evaluation 1/2}
  The implementation is not efficient:
  \begin{itemize}\footnotesize
  \item Batch processing results in error:

    "Known compiler limitation encountered. Sorry."

    Can be circumvented with attribute \texttt{\#[sequential\_outer]}.\pause

  \item It succeeds in generating intra-block version when run with:

    \texttt{\#[incremental\_flattening(only\_intra)]}

    It runs significantly slower for sizes greater than 256 digits.
  \end{itemize}
\end{frame}

\begin{frame}[fragile]
  \frametitle{Division}
  \framesubtitle{Evaluation 2/2}
  Furthermore, runtimes depend on precision rather than size.

  Thus, incomparable to the evaluation method for other arithmetics.\pause

  \phantom{~}

  However, the implementation has no problems compiling to C code.

  Hence, we could use \texttt{multicore} backend and compare to GMP.

  The difference in runtimes are so big that results are meaningless.\pause

  \phantom{~}

  Conclusion: It is inefficient and not comparable to GMP or CGBN.
\end{frame}

%%% Local Variables:
%%% mode: LaTeX
%%% TeX-master: "main"
%%% End:
