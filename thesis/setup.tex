%% PACKAGES

\usepackage[utf8]{inputenc}
\usepackage{babel}
\usepackage{amsmath,amssymb,amsthm,amsfonts,amstext}
\usepackage{mathtools}
\usepackage{geometry}
\usepackage{parskip}
\usepackage{url}
\usepackage{hyperref}
\usepackage[svgnames,table]{xcolor}
\usepackage{color}
\usepackage{graphicx}
\usepackage{stmaryrd}
\usepackage{float}
\usepackage{fancyhdr}
\usepackage{listings}
\usepackage[T1]{fontenc}
\usepackage{fancyhdr}
\usepackage{lipsum}
\usepackage{array}
\usepackage[normalem]{ulem}
\usepackage{nicefrac}

\makeatletter%
\@ifclassloaded{beamer}{}{\usepackage{enumitem}}
\makeatother

%% STYLE

\bibliographystyle{plainurl}

\makeatletter%
\@ifclassloaded{beamer}%
{
  \definecolor{KUrod}{RGB}{144,26,30}
  \usetheme{Copenhagen}
  \setbeamercovered{transparent}
  \setbeamertemplate{headline}{}
  % \useinnertheme{circles}
  \useoutertheme[subsection=false]{miniframes}
  \usecolortheme[named=KUrod]{structure}
}%
{
\renewcommand\@dotsep{280}

\setcounter{tocdepth}{2}

\def\thm@space@setup{\thm@preskip=\parskip \thm@postskip=0pt}

\pagestyle{fancy}
\lhead{\footnotesize Thorbjørn Bülow Bringgaard}
\rhead{\footnotesize Efficient Big Integer Arithmetic Using GPGPU}
\renewcommand{\headrulewidth}{0pt}

\theoremstyle{definition}
\newtheorem{definition}{Definition}[]
\newtheorem{example}{Example}
}%
\makeatother

\newcolumntype{C}[1]{>{\centering\arraybackslash}m{#1}}
\newcolumntype{?}{!{\vrule width 1pt}}

\newfloat{floatenv}{htbp}{lop}
\floatname{floatenv}{Env}
\def\lstfloatautorefname{Env}

%% COMMANDS

\newcommand\cpp{C\texttt{++}}
\newcommand\fun[1]{\ensuremath{\mathtt{#1}}}
\newcommand\tup[2]{\ensuremath{(#1,~#2)}}
\newcommand\arr[1]{\ensuremath{\mathtt{[}#1\mathtt{]}}}
\newcommand{\diagonalarrow}{\ensuremath{\mathrel{\text{\rotatebox[origin=c]{\numexpr30}{$\shortrightarrow$}}}}}
\newcommand{\diagonalarrowdown}{\ensuremath{\mathrel{\text{\rotatebox[origin=c]{\numexpr330}{$\shortrightarrow$}}}}}
\newcommand{\diagonalarrowleftup}{\ensuremath{\mathrel{\text{\rotatebox[origin=c]{\numexpr150}{$\shortrightarrow$}}}}}
\newcommand\bigint[1]{\ensuremath{\mathtt{[}#1\mathtt{]}}}
\newcommand\uint{\texttt{uint}}
\newcommand\red{\color{Crimson}}
\newcommand\cred{\cellcolor{Crimson}}
\newcommand\blue{\color{RoyalBlue}}
\newcommand\cblue{\cellcolor{RoyalBlue}}
\newcommand\green{\color{ForestGreen}}
\newcommand\cgreen{\cellcolor{ForestGreen}}
\newcommand\brown{\color{Chocolate}}
\newcommand\cbrown{\cellcolor{Chocolate}}
\newcommand\steelblue{\color{LightSteelBlue}}
\newcommand\csteelblue{\cellcolor{LightSteelBlue}}
\newcommand\beige{\color{Beige}}
\newcommand\cbeige{\cellcolor{Beige}}
\newcommand\gray{\color{LightGray}}
\newcommand\cgray{\cellcolor{LightGray}}
\newcommand\purple{\color{Plum}}
\newcommand\cpurple{\cellcolor{Plum}}

%% LISTINGS

\lstdefinelanguage{futhark}
{
  % list of keywords
  morekeywords={
    do,
    else,
    for,
    if,
    in,
    include,
    let,
    loop,
    then,
    type,
    val,
    while,
    with,
    module,
    def,
    entry,
    local,
    open,
    import,
    assert,
    match,
    case,
  },
  sensitive=true, % Keywords are case sensitive.
  morecomment=[l]{--}, % l is for line comment.
  morestring=[b]" % Strings are enclosed in double quotes.
}

\lstdefinelanguage{CPP}
{
  language = C++,
  morekeywords = {
    uint32_t,
    uint64_t,
    uint_t,
    carry_t,
    __syncthreads,
    Base,
    dim3,
    CarryProp,
    ubig_t,
    quint_t,
    SegCarryProp,
    B,
  },
}

\lstdefinelanguage{pseudo}
{
  % list of keywords
  morekeywords={
    do,
    else,
    for,
    if,
    in,
    then,
    while,
    map,
    map2,
    scan_exc,
    shift,
    fun,
    return,
    concat,
    reverse,
    min,
    max,
    take
  },
  sensitive=true,
  morecomment=[l]{--},
}

\lstdefinestyle{myStyle}
{
  frame=tb,
  basicstyle=\footnotesize\ttfamily,
  keywordstyle = \color{RoyalBlue},
  commentstyle = \color{ForestGreen},
  stringstyle = \color{purple},
  numberstyle=\scriptsize\color{darkgray},
  captionpos=t,
  breaklines=true,
  numbers=left,
}

\lstset{style=myStyle}

%%% Local Variables:
%%% mode: LaTeX
%%% TeX-master: "main"
%%% End:
