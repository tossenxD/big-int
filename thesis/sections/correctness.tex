\section{Correctness}
\label{sec:cor}

Our big integer arithmetic operators are defined to be exact and efficient. The
former property collapses under incorrectness and more, the latter loose its
meaning. Performance benchmarks of invalid implementations are not trustworthy,
as the underlying algorithms may not be fully represented, and therefore,
correctness is a prerequisite of measuring the efficiency of our
implementations. Additionally, we have build a tower of arithmetics, and so the
properties of the whole tower collapses under an invalid operator.

While proofs are given by the authors of the algorithms, this section focuses on
testing the correctness of the implementations. It is not plausible to
exhaustively test the correctness on all big integers, and we might leave out
important input patterns by writing manual test cases. Hence, we test on
randomly generated big integers and compare the result of running our
implementations against GMP's, assuming that GMP gives the valid answer.

\paragraph{Testing Methodology}
Testing the results of our CUDA implementations against GMP's is
straightforward, because GMP can be linked to C++ and called directly in from
our host function. We generate some random big integers on which we call both
ours and GMP's arithmetic functions on and compare the results. To generate a
random big integer $u$ of size $m$ and $\mathit{nz}$ non-zero digits, we use the
code in Listing \ref{brngcpp}:
\begin{lstlisting}[language=CPP, caption={\footnotesize Random big integer generator in C++ with $u$ of size $m$ and $nz$ non-zero digits.}, label={brngcpp}]
for(int i = 0; i < m; i++) {
    uint32_t x = 0;
    if(i < nz) {
        uint32_t low  = rand()*2;
        uint32_t high = rand()*2;
        x = (high << 16) + low;
    }
    u[i] = x;
}
\end{lstlisting}

Regarding Futhark, the compiler has a builtin test functionality called
\texttt{futhark-test} \cite{futguide}. It allows us to write test-specifications
in Futhark programs that the compiler can generate and run. A specification
consist of a set of inputs and their the expected outputs. The inputs can be
either randomly generated or fixed, and the outputs can be either fixed or
automatically determined (by assuming C as backend gives the correct result).

It is not possible to directly test our Futhark implementations against GMP's
through \texttt{futhark-test}, but it is possible indirectly by constructing the
tests in a peculiar way: We write a small Futhark library, called
\texttt{gmp-validation-lib.fut}, which exports exactly one of each arithmetic
function. Each function takes exactly two inputs arrays and produce exactly one
output array, while being agnostic to the shape of the input. I.e.\ for the
underlying arithmetics that requires the input to be a specific shape (read;
\textit{convmul}), the library function adds padding to ensure that the shapes
matches. Now, we compile this library to a C API using the Futhark library
feature \cite{futguide}.

This produces a C header file that exports the functions and data types of the
compiled library. In turn, we write a C-program that incorporates GMP as
described above, and tests the Futhark arithmetic functions from the C-program
through the API.

Hence, we obtain a GMP validated \textit{black box} Futhark library. We can
therefore substitute GMP function calls in Futhark test programs with function
calls to this library, effectively giving us access to GMP inside the
\texttt{futhark-test} environment.

The test specifications assign randomly generated 2D arrays of different sizes
as inputs and then expects the output to be \texttt{true}. The test function
that compares our addition functions against GMP is shown in Listing
\ref{futaddvaltest} below -- an identical test function is defined for
multiplication. Here, \texttt{test\_add} refers to the GMP-validated addition
and \texttt{oneAddVX} to version \texttt{X} of our addition
implementations. Line 24 defines a validation function, lines 25-30 run our
addition implementations, and line 31 checks that all results validate.

\lstinputlisting[language=futhark,firstline=22,lastline=31,caption={\footnotesize Futhark testing-functions for addition with 64-bit base from \texttt{fut-validation.fut}.},label={futaddvaltest},firstnumber=22]{../futhark/test/fut-validation-main.fut}

Similarly to the definition of validation tests, we also define property-based
tests over the usual arithmetic properties, such as commutativity for
addition. They are included for the purpose of completeness (and a potential
source for debugging information), but are redundant w.r.t.\ the GMP validation,
assuming GMP gives the correct result.

(A final note on validating division in C; the divisor has a randomly generated
$nz$ value to avoid hitting the same special cases repeatedly, providing more
meaningful tests.)

\paragraph{Test Results}
We run the Futhark tests using \texttt{futhark-test} on base \texttt{u16},
\texttt{u32}, and \texttt{u64} integers in the size range $4\leq m \leq 512$ and
$num\_instances = 2048$, with $ipb = 4$ where relevant. We run all tests twice
with different backends, once on the CPU with C as backend, and once on the GPU
with OpenCL as backend. The tests includes properties and validation against
GMP, and they run on all four versions of \textit{badd} (addition) and all three
versions of \textit{convmul} (multiplication). \uline{All tests validates}.

Next is the Futhark-GMP validation in C. We test addition, subtraction,
multiplication, and division. We run the tests of base \texttt{uint16\_t},
\texttt{uint32\_t}, and \texttt{uint64\_t}, with division only ran on
\texttt{uint16\_t}. They run on two different integer size patterns: 350 tests
with $m=\{2,4,8,16,32,64,128\}$, 50 tests for each size, and 2500 tests with 50
randomly generated sizes in the range $1\leq m \leq 512$, 50 tests for each size. The
tests are run twice using the Futhark functions compiled to C and OpenCL,
respectively. \uline{Tests for addition, subtraction, and multiplication
  validates -- tests for division does not validate}.

Lastly, we have the CUDA-GMP validations tests. We test all three versions of
\textit{badd} and all five versions of \textit{convmul} with base
\texttt{uint32\_t} and \texttt{uint64\_t}. The test setup is fused with the
performance benchmark setup (i.e.\ after a benchmark has been run, the result
gets valid), so further details about the test-configuration are in section
\ref{subsec:benchset}. \uline{All tests validate}.

%%% Local Variables:
%%% mode: LaTeX
%%% TeX-master: "../main"
%%% End:
