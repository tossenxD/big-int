\section{Correctness}
\label{sec:cor}

A big integer library is only as useful as it is correct. Furthermore, we cannot
trust performance benchmarks of incorrect implementations, as the underlying
algorithms may not be fully implemented.

While proofs are given in earlier sections or by the authors of the algorithms,
this section focuses on testing the correctness of the implementations.

It is not plausible to test correctness for all big integers, and we might leave
out important input patterns by writing manual test cases. Instead, we test on
randomly generated big integers and compare the result of running our
implementations against GMP's, assuming that GMP gives the correct result.

\subsection{Testing Methodology}

Testing the correctness of CUDA functions using GMP is straightforward, because
the functions are already implemented in C++. We simply generate some random big
integers, call both ours and GMP's functions and compare the results. To
generate a random big integer $u$ of precision $m$ and $\mathit{nz}$ non-zero
digits, we use Listing \ref{brngcpp}.

\begin{lstlisting}[language=CPP, caption={Random big integer generator in C++.}, label={brngcpp}]
for(int i = 0; i < m; i++) {
    uint32_t x = 0;
    if(i < nz) {
        uint32_t low  = rand()*2;
        uint32_t high = rand()*2;
        x = (high << 16) + low;
    }
    u[i] = x;
}
\end{lstlisting}

The Futhark compiler comes with a builtin testing functionality called
\texttt{futhark-test} \cite{futguide}. It allows us to write test-programs in
Futhark, specifying test-inputs with their the expected output for functions
defined in the Futhark program. The inputs can be both randomly generated or
fixed, and the output can be either fixed or automatically determined by
compiling the function to C.

However, we cannot import GMP to a Futhark test-program. Instead, we can write a
C-program utilizing GMP as described above, and test our arithmetic functions
from the C-program through an API by compiling the Futhark functions to a C
library.

{\color{red} TODO write about C API under preliminaries}

The file \texttt{gmp-validation-lib.fut} exports exactly one version of each
arithmetic function in both 32bit and 64bit base. We compare the exported
functions result against GMP on randomly generated input in the file
\texttt{gmp-validation.c}. For the sake of simplicity, the library exports the
fundamental versions (v1's), which also allows us to test only on
single-instance big integer inputs.

The remaining versions are tested against the GMP-validated one in the Futhark
test-program \texttt{fut-validation.fut} by using \texttt{futhark-test}. We test
both on single-instance and multi-instance big integer inputs. E.g. Listing
\ref{futaddvaltest} contains the testing-functions for addition of 64bit base
with GMP-validated library function \fun{test\_add64}.

\lstinputlisting[language=futhark,firstline=112,lastline=127,caption={\footnotesize Futhark testing-functions for addition with 64-bit base from \texttt{fut-validation.fut}.},label={futaddvaltest},firstnumber=112]{../futhark/test/fut-validation.fut}

Lastly, we also test that the usual arithmetic properties hold for their
respective operator, i.e. associativity, commutativity, distributivity,
etc.. These properties are tested using \texttt{futhark-test} in the file
\texttt{fut-properties.fut} and are tested independent of GMP. E.g. Listing
\ref{futaddproptest} contains commutativity test for 64-bit base addition.

\lstinputlisting[language=futhark,firstline=169,lastline=174,caption={\footnotesize Futhark commutativity test for addition with 64-bit base from \texttt{fut-properties.fut}.},label={futaddproptest},firstnumber=169]{../futhark/test/fut-properties.fut}

{\color{red} TODO testing div with random $nz$ divisor to ensure meaningful tests}

\subsection{Test Results}


%%% Local Variables:
%%% mode: LaTeX
%%% TeX-master: "../main"
%%% End:
