\section{Preliminaries}
\label{sec:pre}
Common processors in a computer includes a Central Processing Unit (CPU) and a
Graphics Processing Unit (GPU). The CPU typically executes general purpose
routines, while the GPU is specialized in massive parallel processing of the
numerous vectors that constitutes computer graphics. However, the GPU is not
confined to processing graphics and can act as a General Purpose GPU (GPGPU) to
execute general parallel routines.

Compared to the CPU, GPGPU has the benefits of higher instruction throughput and
higher bandwidth (achieved by various design differences such as increased core
count), but has the main drawback of requiring substantially more effort to
properly utilize. For our purposes, GPGPU provides a significant performance
speedup and scalability in exchange for parallel algorithms and programs of
greater complexity.

This section is structured as follows: In \ref{subsec:gpgpu} we introduce the
GPGPU architecture and parallel execution model. In \ref{subsec:cuda} we give an
overview of the CUDA \cpp\ environment, which allows interfacing with low-level
GPU primitives in the high-level language \cpp. In \ref{subsec:futhark} we give
an overview of the high-level programming language Futhark that can compile to
GPU executable code without the need to manually manage low-level primitives.

\subsection{GPGPU Architecture}
\label{subsec:gpgpu}

There are two main interfaces for a GPU: OpenCL and CUDA, the former being an
API for an open set of standards and instructions, and the latter a proprietary
platform for NVIDIA GPUs consisting of specialized instructions. This thesis
focus on CUDA, yet the algorithmic aspects are kept general and allows for
OpenCL operability, along with the software developed in Futhark compiling to
both models.

GPGPU is parallel by design and consists of up to thousands of multithreaded
cores. The architecture, according to NVIDIA's own CUDA \cpp\ programming guide
\cite{cudaguide}, is as follows: Threads are grouped in 32, where groups are
called \textit{warps}. A warp executes one instruction at a time in lockstep,
called Single Instruction, Multiple Threads (SIMT). A program, called a
\textit{kernel}, is executed on a \textit{grid} of \textit{blocks} (also called
a thread block) with a specified amount of threads per block. The number of
threads in a block (block size) and the number of blocks in a grid (block count)
must be statically known prior to kernel execution, with a maximum block size of
1024 threads. Each warp, block, and grid has its own memory called
\textit{local}, \textit{shared}, and \textit{global} memory, respectively, with
the former being the smallest and fastest, and the latter the largest and
slowest.

SIMT allows writing parallel programs of minimal overhead w.r.t.\
synchronization and memory latency. It urges to keep procedures closest to
warp-level, utilizing faster memory, lockstep execution, and optimizing for
locality of reference within a warp. This introduce the idea of
\textit{coalesced memory accesses}, where consecutive threads access consecutive
memory locations, resulting in as few memory transactions as possible, and
necessary for maintaining high bandwidth while accessing global memory. However,
consecutive threads may not be supposed to access consecutive elements. In such
case, it is faster to use shared memory as a buffer s.t.\ transactions between
shared and global memory are coalesced, and the original uncoalesced
transactions are between shared and local memory. Figure \ref{fig:coalesced}
illustrates this approach with an example. While we may not gain much speed in
the shown example, it becomes a significant latency reduction with many more
threads and elements per threads.

\begin{figure} \centering
  { \centering \small
    \begin{tabular}{C{0.03\textwidth}C{0.45\textwidth}C{0.45\textwidth}}
      & \footnotesize Consecutive threads reads pairwise elements from global memory & \footnotesize Consecutive threads 1.\ reads consecutive from global memory, 2.\ writes consecutive to shared memory, 3.\ read pairwise from shared memory.
      \end{tabular}\\[0.5ex]
    \begin{tabular}{C{0.0275\textwidth}?C{0.45\textwidth}|C{0.45\textwidth}}
      $G_{\mathit{id}}$ &
      \begin{tabular}{|C{0.4cm}|C{0.4cm}|C{0.4cm}|C{0.4cm}|C{0.4cm}|C{0.4cm}|C{0.4cm}|C{0.4cm}|}
\hline $g_0$ & $g_1$ & $g_2$ & $g_{3}$ & $g_{4}$ & $g_{5}$ & $g_6$ & $g_7$\\
\hline
      \end{tabular} &
      \begin{tabular}{|C{0.4cm}|C{0.4cm}|C{0.4cm}|C{0.4cm}|C{0.4cm}|C{0.4cm}|C{0.4cm}|C{0.4cm}|}
\hline $g_0$ & $g_1$ & $g_2$ & $g_{3}$ & $g_{4}$ & $g_{5}$ & $g_6$ & $g_7$\\
\hline
      \end{tabular}\\
      &
\begin{tabular}{C{0.003cm}C{0.003cm}C{0.42cm}C{0.003cm}C{0.003cm}C{0.42cm}C{0.003cm}C{0.003cm}C{0.4cm}C{0.003cm}C{0.003cm}C{0.12cm}}
$\shortuparrow_{r}$ & $\diagonalarrow$ & & $\shortuparrow_{r}$ &
$\diagonalarrow$ & & $\shortuparrow_{r}$ & $\diagonalarrow$ & &
$\shortuparrow_{r}$ & $\diagonalarrow$ & \\
      \end{tabular} &
      \begin{tabular}{C{0.41cm}C{0.41cm}C{0.41cm}C{0.41cm}C{0.41cm}C{0.41cm}C{0.41cm}C{0.41cm}}
$~\shortuparrow_{r}$ & $~\shortuparrow_{r}$ & $~\shortuparrow_{r}$ &
$~\shortuparrow_{r}$ & $~\shortuparrow_{r}$ & $~\shortuparrow_{r}$ &
$~\shortuparrow_{r}$ & $~\shortuparrow_{r}$ \\
      \end{tabular}\\[-0.7ex]
      $T_{\mathit{id}}$ &
\begin{tabular}{C{0.41cm}C{0.41cm}C{0.41cm}C{0.41cm}C{0.41cm}C{0.41cm}C{0.41cm}C{0.41cm}}
$~t_0$ & & $~t_1$ & & $~t_{2}$ & & $~t_3$ & \\
      \end{tabular} &
      \begin{tabular}{C{0.41cm}C{0.41cm}C{0.41cm}C{0.41cm}C{0.41cm}C{0.41cm}C{0.41cm}C{0.41cm}}
$~t_0$ & $~t_1$ & $~t_2$ & $~t_3$ & $~t_{0}$ & $~t_1$ & $~t_2$ & $~t_3$ \\
      \end{tabular}\\[-0.7ex]
       & &
      \begin{tabular}{C{0.41cm}C{0.41cm}C{0.41cm}C{0.41cm}C{0.41cm}C{0.41cm}C{0.41cm}C{0.41cm}}
$~\shortdownarrow_{w}$ & $~\shortdownarrow_{w}$ & $~\shortdownarrow_{w}$ &
$~\shortdownarrow_{w}$ & $~\shortdownarrow_{w}$ & $~\shortdownarrow_{w}$ &
$~\shortdownarrow_{w}$ & $~\shortdownarrow_{w}$ \\
      \end{tabular}\\
      $S_{\mathit{id}}$ & &
      \begin{tabular}{|C{0.4cm}|C{0.4cm}|C{0.4cm}|C{0.4cm}|C{0.4cm}|C{0.4cm}|C{0.4cm}|C{0.4cm}|}
\hline $s_0$ & $s_1$ & $s_2$ & $s_{3}$ & $s_{4}$ & $s_{5}$ & $s_6$ & $s_7$\\
\hline
      \end{tabular}\\
      & &
      \begin{tabular}{C{0.003cm}C{0.003cm}C{0.42cm}C{0.003cm}C{0.003cm}C{0.42cm}C{0.003cm}C{0.003cm}C{0.4cm}C{0.003cm}C{0.003cm}C{0.12cm}}
$\shortuparrow_{r}$ & $\diagonalarrow$ & & $\shortuparrow_{r}$ &
$\diagonalarrow$ & & $\shortuparrow_{r}$ & $\diagonalarrow$ & &
$\shortuparrow_{r}$ & $\diagonalarrow$ & \\
      \end{tabular}\\[-0.7ex]
       $T_{\mathit{id}}$ & &
      \begin{tabular}{C{0.41cm}C{0.41cm}C{0.41cm}C{0.41cm}C{0.41cm}C{0.41cm}C{0.41cm}C{0.41cm}}
$~t_0$ & & $~t_1$ & & $~t_{2}$ & & $~t_3$ & \\
      \end{tabular}\\
    \end{tabular}
  }
  \caption{\footnotesize Example of coalesced global memory access using a
    shared memory buffer -- $G$ denotes global memory cells, $S$ shared memory
    cells, $T$ the threads, and $r$-$w$ indicates read-write
    transactions. Left-hand-side shows uncoalesced global memory access, while
    right-hand-side shows coalesced access in 3 steps.}
    \label{fig:coalesced}
  \end{figure}


  An important side-effect of SIMT is no divergent control flow (branching)
  within warps. I.e.\ suppose only 1 thread in a warp choose an \fun{if}-branch;
  then, the remaining 31 threads cannot proceed execution before the first
  thread has finished, and instead, executes the \texttt{if}-branch instructions
  on dummy data.

  Thus, in order to construct efficient programs for GPGPU that exhibit high
  throughput and bandwidth, they must be inherently parallel, contain minimal
  amount of branching, use fast memory, run as close to warp-level as possible,
  and access global memory coalesced.

\subsection{The CUDA \cpp\ Programming Interface}
\label{subsec:cuda}

In this thesis, we interface with CUDA by means of a \cpp\ language extension
that compiles a program to both a CPU and GPU executable portion using C and the
CUDA driver API, as described in the CUDA \cpp\ guide \cite{cudaguide}. The CPU
running the initial program is called the \textit{host} and the target GPU is
called the \textit{device}. The extension exports functions in the
\textit{alloc}-family, that allows the host to preallocate, read, and write
memory of the device. Device functions (i.e.\ kernels) are called from the host,
and take parameters in the form of pointers to memory pre-transferred to the
device.

The extension exports some function execution- and variable memory- space
specifiers. These infer to the compiler how to treat functions and memory (e.g.\
host, kernel, or device function, shared or local memory, etc.). The extension
also exports some device functions, notably the function
\texttt{\_\_syncthreads()}. It creates a barrier preventing threads from further
code execution until all threads within the block has reached that barrier.

In order to write parametric kernels in e.g.\ the type or size of big integers,
we can use \cpp\ templates. They provide generic programming over type and value
parameters that are concretized at compile time, as written by Stroustrup in
\cite{stroustrup}. Templates especially prove to be a strong tool when combined
with other high-level \cpp\ constructs, such as classes.  (On a side note, the
CUDA design has no explicit error handling on the device. Instead, CUDA API
calls returns error codes, which the host must then explicitly check and
handle.)

Lastly, threads are implicit: Kernels launch with an \textit{execution
  configuration} of the form ``\fun{ker<<< grid, block >>>(params)}'', where
\fun{grid} and \fun{block} specifies the grid and block dimensions,
respectively, and are of type \fun{dim3}, denoting the specifications in 3D. The
spawned kernel has access to a set of special objects, notably \fun{threadIdx},
\fun{blockIdx}, and \fun{blockDim} containing the index of the currently
executing thread, the index of the block that it belongs to, and block size,
respectively. Thus, the executable kernel code is run on each of the specified
threads, and spatial awareness within a thread is achieved explicitly through
the special objects -- as demonstrated in the following example:

\begin{example}[a simple CUDA \cpp\ program]\label{ex:cuda}
  The following program is a simple example of the general routines
  involved with parallel programming through the CUDA \cpp\ extension.
\begin{lstlisting}[language=CPP,escapeinside={(*}{*)}]
(*{\color{Crimson}$\bullet$ \texttt{Kernel function that increments $m$ integers of the input array by one.}*)
template<int m> __global__ void incrKernel(int* input, int* output) {
  // find the global thread ID of the currently executing thread
  const unsigned int global_id = blockDim.x*blockIdx.x + threadIdx.x;
  // if its ID does not exceed the input size, it executes the operation
  if (global_id < m)
    output[global_id] = input[global_id] + 1;
}
(*{\color{Crimson}$\bullet$ \texttt{Host function to execute the above kernel assuming array $\mathit{h\_mem}$ contains $m$ integers.}*)
...
  // allocate and transfer memory to device
  int *d_mem_in, *d_mem_out;
  cudaMalloc((void**) &d_mem, m * sizeof(int));
  cudaMemcpy(d_mem_in, h_mem, m * sizeof(int), cudaMemcpyHostToDevice);
  // execute the kernel below with (*\green $\lceil m/512 \rceil$*) blocks of 512 threads each
  dim3 block (512, 1, 1);
  dim3 grid (1 + ((m - 1)/512), 1, 1);
  incrKernel<m><<< block, grid >>>(d_mem_in, d_mem_out);
  // fetch results
  cudaMemcpy(h_mem, d_mem_out, m * sizeof(int), cudaMemcpyDeviceToHost);
...
\end{lstlisting}
\end{example}

From now on, we denote a CUDA \cpp\ program simply as a CUDA program and assume:
\begin{itemize}
  \renewcommand\labelitemi{--}
\item Kernel dimensions to be 1D (as they are in Example \ref{ex:cuda} above).
\item The existence of sensible host functions to execute and error-check
  kernels.
\item A small library of common subroutines such as accessing global memory
  coalesced.
\end{itemize}
These assumptions allow us to focus on the implementation of algorithms.

\subsection{The Futhark Programming Language}
\label{subsec:futhark}

Futhark is a high-level pure functional language with emphasis on data parallel
array programming
\cite{ParallelProgrammingInFuthark,Henriksen:2017:FPF:3062341.3062354}. It is
hardware-agnostic and can compile to both sequential or parallel C, OpenCL, and
CUDA code \cite{futguide}. The fundamental design of Futhark revolves around
\textit{Second-Order Array Combinators} (SOACs). They are array-functions that
are easy to reason about and define in terms sequential semantics, while still
being able to compile to (efficient) parallel code.

Analyzing the asymptotics of parallel programs is separated into \textit{work}
and \textit{depth}. Work refers to the total amount of computations in the
program -- known as the traditional basis of time complexity analysis. However,
in the parallel domain, work is distributed amongst threads, and so it may not
be an accurate representation of runtimes. The depth (also called span) is the
amount of sequential work within a thread, given infinitely many threads.

We now present the most fundamental SOACs, which we use throughout this thesis
to analyse parallelism.\footnote{Other combinators exists, but these are the
  most crucial to this thesis. An overview of the array combinators and
  functions can be found in the Futhark standard library \cite{futprelude}.} We
assume their function inputs are $O(F_{w}(n))$ and $O(F_d(n))$.

First we have the combinator \fun{map} of work $O(n\cdot F_w(n))$ and depth
$O(F_d(n))$. It distributes a function over an array (known from other
functional languages such as Haskell \cite{marlow2010haskell}) and can
inherently be executed in parallel. The type signature and semantics is:
\begin{align}
  &\mathtt{map}:(\tau \to \tau') \to \arr{\tau} \to \arr{\tau'}\\
  &\mathtt{map}~f~[a_0,~a_1,\ldots,~a_{n-1}] \coloneq [f~a_0,~f~a_1,\ldots,~f~a_{n-1}]
\end{align}

Next we have the combinator \fun{reduce} of work $O(n\cdot F_w(n))$ and depth
$O(\log n \cdot F_w(n))$.  The semantics is akin to a \fun{fold} from other
functional languages, but with one big difference: The operator must be
associative and have a left-associative neutral element, allowing the array to
be accumulated in $O(\log n)$ steps. The type signature and semantics is:
\begin{align}
  &\mathtt{reduce}:(\tau \to \tau \to \tau) \to \tau \to \arr{\tau} \to \tau\\
  & \mathtt{reduce} \oslash e~[a_0,~a_1,\ldots,~a_{n-1}] \coloneq e \oslash a_0 \oslash a_1 \oslash \ldots\oslash a_{n-1}
\end{align}

Similarly, we have the combinator \fun{scan} with work $O(n\cdot F_w(n))$ and depth
$O(\log n \cdot F_w(n))$. The semantics corresponds to an accumulated reduction over
the input array (also called a prefix sum), and so the associativity and neutral
element restrictions of the operator applies as well. The type signature and
semantics is:
\begin{align}
  &\mathtt{scan}:(\tau \to \tau \to \tau) \to \tau \to \arr{\tau} \to \arr{\tau}\\
  & \mathtt{scan} \oslash e~[a_0,~a_1,\ldots, ~a_{n-1}] \coloneq [e \oslash a_0,~e \oslash a_0 \oslash a_1,\ldots,~e \oslash a_0 \oslash a_1 \oslash \ldots\oslash a_{n-1}]
\end{align}

Lastly we have the combinator \fun{scatter} with work $O(n)$ and depth $O(1)$.
It takes a destination, index, and value array, and performs an in-place
distribution of the values over the destination array according to the
indices. The indices and values must have same shape. Values are ignored for
indices that are out-of-bound. The type signature is:
\begin{equation}
\mathtt{scatter}:\arr{\tau} \to \arr{\mathtt{i64}} \to \arr{\tau} \to \arr{\tau}
\end{equation}

One of the strengths of the Futhark compiler is its ability to \textit{fuse}
chains of \texttt{map}-\texttt{reduce} compositions. This allow us to write nice
and clean code, straightforwardly model parallel algorithms, and let the
compiler generate more complex and optimized GPGPU code that use less
intermediate values and instructions.

While Futhark is more clean and succinct than CUDA, it contains compromises that
allows Futhark to compile to GPGPU code, making it restricted in comparison to
other high-level functional languages such as Haskell. E.g.\ Futhark allows
nested arrays as data structure, but all data must be flat on a GPGPU. Hence,
the nested arrays must be regular in order for the compiler to know how to
flatten them for execution. The shape of arrays must be statically inferred to
the compiler too -- primarily part of the type signature of functions, but can be
manually inferred. Example \ref{ex:fut} shows the difference in CUDA and Futhark
clearly:

\begin{example}[a simply Futhark program]\label{ex:fut}
  The following program is a Futhark translation of the CUDA program of Example
  \ref{ex:cuda} that increments the integers in a given array by one:
\begin{lstlisting}[language=futhark]
def incrFut [m] (input: [m]i32) : [m]i32 = map (+ 1) input
\end{lstlisting}
\end{example}

Overall, Futhark is meant to balance usability, power of abstractions, and
parallel efficiency.

%%% Local Variables:
%%% mode: LaTeX
%%% TeX-master: "../main"
%%% End:
