\section{Software Structure}
\label{sec:software}

Various big integer arithmetic implementations has been developed as part of
this Thesis. The code is publicly available at the GitHub repository
\url{https://github.com/tossenxD/big-int}, where the latest commit of writing
has ID {\red <INSERT LATEST COMMIT>}.

The repository is structured as follows: The directories \texttt{cuda},
\texttt{futhark}, and \texttt{prototypes} contains CUDA, Futhark, and C code,
respectively. Each directory includes a \texttt{README} with detailed
explanations of the setup and a \texttt{Makefile} to replicate the results of
this Thesis.

\paragraph{CUDA}
The files \texttt{ker-add.cu.h} and \texttt{ker-mul.cu.h} contains the kernels
for addition and multiplication. The arithmetics are gradually optimized over
three and five versions, respectively, kept in the files for the sake of
comparison. The kernels are called from the main file \texttt{main.cu}, which
serves to generate inputs, check for errors, test the arithmetics, and run
benchmarks. The heading of the main file defines numerous parameters specifying
integer base, kernel version, what to run, etc., which can manually be adjusted.

The \texttt{Makefile} offers two ways to run the main file -- with either a
normal or small amount of total work. They are invoked by \texttt{\$ make} and
\texttt{\$ make small}, respectively.

Files to run CGBN is in subdirectory \texttt{cgbn}, but have a similar setup to
the primary files.

\paragraph{Futhark}
The files \texttt{add.fut}, \texttt{sub.fut}, \texttt{mul.fut}, and
\texttt{div.fut} contains Futhark programs with big integer addition,
subtraction, multiplication, and division. The addition and multiplication has four
and three versions, respectively, similarly to the CUDA kernels. The big integer
base type can be configured in the file \texttt{helper.fut}. While the programs
will compile for all base types, the division program is nonsensical for bases
other than \texttt{u16}.

The subdirectories \texttt{test} and \texttt{bench} contains the Futhark test
and benchmark programs. Benchmarks are straightforward, and can be run by
\texttt{\$ make add-uX} or \texttt{\$ make mul-uX}, where \texttt{X} is the bits
of the specified base type (64 is default and 16 is not supported).

The tests includes a C file that is tricky to call -- using the \texttt{Makefile}
is recommended. They can be run with \texttt{\$ make test-uX}, where \texttt{X}
is the bits of the base type (64 is default).

\paragraph{Prototype}
The file \texttt{div.c} contains a sequential prototype for big integer
division, that may be of interest to run experiments. It can be called either
with random or fixed inputs as \texttt{\$ make random} or \texttt{\$ make
  fixed}. Both the number of random inputs and the form of the fixed inputs can
be adjusted in the \texttt{Makefile}. A 1000 random inputs is default.

%%% Local Variables:
%%% mode: LaTeX
%%% TeX-master: "../main"
%%% End:
