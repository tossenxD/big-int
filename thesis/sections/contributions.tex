\section{Software Structure}
\label{sec:cont}

We have produced multiple big integer arithmetic implementations in the
development of this Thesis. The code is publicly available at the GitHub
repository \url{https://github.com/tossenxD/big-int}, where the latest commit,
as of writing, has ID {\red <INSERT LATEST COMMIT>}.

The repository is structured as follows: The directories \texttt{cuda},
\texttt{futhark}, and \texttt{prototypes} contains CUDA, Futhark, and C code,
respectively. In each directory there is a \texttt{README} and \texttt{Makefile}
to replicate the results of this Thesis.

The files \texttt{add.fut} and \texttt{ker-add.cu.h} contains implementations of
big integer addition. They have four and three versions, respectively, that are
optimized in varying degrees. We found one of the versions to be strictly better
performing. However, in Futhark, we found it ambiguous which version is
performing best when fusing multiple consecutive additions.

The files \texttt{mul.fut} and \texttt{ker-mul.cu.h} contains implementations of
classical big integer multiplication. They define three and five versions,
respectively, whence two versions are performing best based on the size of the
integers. These implementations also reveals a performance gap between Futhark
and CUDA, and what optimizations even apply in the two languages.

Lastly, the files \texttt{div.c} and \texttt{div.fut} contains implementations
of exact big integer division by whole shifted inverse. The C implementation is
a sequential adaptation of the algorithm proposed by Watt in
\cite{watt2023efficient}, wherefrom we parallelize in the Futhark
implementation. These are not performance-optimized in the same manner as
addition and multiplication, but moreso serves as a proof of concept of a
parallel implementation of the algorithm. During the development we found some
valuable insight, notably a corner case unconsidered in the original formulation
of the algorithm. Hence, we also present a revised algorithm specialized for big
integers and parallel execution.

{\red [describe how to run code]}

%%% Local Variables:
%%% mode: LaTeX
%%% TeX-master: "../main"
%%% End:
