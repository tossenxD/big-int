\section{Multiplication}
\label{sec:mul}

Multiplication is the next of the basic arithmetic operators. It is more
complicated than addition, since all digits of the multiplicand have to be
multiplied by all digits of the multiplier, leading to assymptoticly quadratic
runtimes. Multiple assymptotically faster algorithms exists, such as Karatsuba
and Fast Fourier Transform (FFT), both of which are mentioned by Knuth and part
of GMP \cite{knuth97} \cite{GMP}.

In section \ref{subsec:mulalg}, we present the classical multiplication
algorithm that convolutes the digits of the operands, and discuss how to
parallelize it. We also discuss the Karatsuba and FFT algorithm, where
especially the FFT is promising in a GPGPU setting. In sections
\ref{subsec:mulcud} and \ref{subsec:mulfut} we present our CUDA and Futhark
implementations, respectively, and any further optimizations related to the
implementations. Lastly, in section \ref{subsec:mulsin} we go over the special
case of multiplication by a single precision factor and present an algorithm to
handle it.

\subsection{Algorithm}
\label{subsec:mulalg}

In this section, we first assess the \textit{classical} multiplication algortihm
in section \ref{subsubsec:classical}, before we move on to more advanced
algorithms in section \ref{subsubsec:fft}.

\subsubsection{Classical}
\label{subsubsec:classical}
The classical multiplication multiplication algorithm handles one digit of the
multiplier at a time. E.g. in decimal sytem, $42\cdot 21$ becomes
$1 \cdot 42 + 2 \cdot 42 \cdot 10$. In the positional number system of our big integers, the
definition of classical multiplication is:
\begin{equation}
  \label{eq:clasmul0}
  u \cdot v = \sum_{i=0}^{m-1}u_i\cdot \left( \sum_{j=0}^{m-1}v_j\cdot B^{j} \right)B^{i}
\end{equation}
Two observations: We cannot store the inner sum in a single word and we only
care about the first $m$ digits of the result\footnote{The result of
  multiplication is up to twice the size of the inputs. In order to not loose
  precision in our setting, a memory region of double size should be
  preallocated before the multiplication function call.}.  Hence, we can define
it making use of tiling, as illustrated by Figure \ref{fig:tiledmult}, s.t. the
inner multiplications can make use of native hardware arithmetics:

{
\begin{minipage}{0.5\textwidth}
  \centering
  \begin{figure}[H]
  \centering
  \footnotesize
  \begin{tabular}{c}
    \begin{tabular}{|C{0.8cm}|C{0.8cm}|C{0.8cm}|C{0.5cm}|C{0.8cm}|}
      \hline
      $w_0$ & $w_1$ & $w_2$ & $\ldots$ & $w_{m-1}$\\
      \hline
    \end{tabular}\\
    \begin{tabular}{C{0.8cm}C{0.8cm}C{0.8cm}C{0.5cm}C{0.8cm}}
      $=$ & $=$ & $=$ &  & $=$
    \end{tabular}\\[-2ex]
    \begin{tabular}{C{0.2cm}C{0.2cm}C{0.2cm}C{0.2cm}C{0.2cm}C{0cm}C{0.7cm}C{0.2cm}C{0.2cm}C{0.2cm}C{0.2cm}}
      & & $\diagonalarrowdown$ & & $\diagonalarrowdown$ & & $\diagonalarrowdown$ & $\diagonalarrowdown$ & &
    \end{tabular}\\[-1.4ex]
    \begin{tabular}{C{0.8cm}C{0.8cm}C{0.8cm}C{0.5cm}C{0.8cm}}
      $0$ & $c_0$ & $c_1$ & $\ldots$ & $c_{m-2}$
    \end{tabular}\\[-0.5ex]
    \begin{tabular}{C{0.8cm}C{0.8cm}C{0.8cm}C{0.5cm}C{0.8cm}}
      $+$ & $+$ & $+$ &  & $+$
    \end{tabular}\\[-0.5ex]
    \begin{tabular}{C{0.8cm}C{0.8cm}C{0.8cm}C{0.5cm}C{0.8cm}}
      $u_0 v_0$ & $u_0 v_1$ & $u_0 v_2$ & $\ldots$ & $u_0v_{m-1}$
    \end{tabular}\\[-0.5ex]
    \begin{tabular}{C{0.8cm}C{0.8cm}C{0.8cm}C{0.5cm}C{0.8cm}}
       & $+$ & $+$ &  & $+$
    \end{tabular}\\[-0.5ex]
    \begin{tabular}{C{0cm}C{0.8cm}C{0.8cm}C{0.8cm}C{0.5cm}C{0.8cm}C{0cm}}
       & & $u_1v_0$ & $u_1v_1$ & $\ldots$ & $u_1v_{m-2}$ & $\ldots$
    \end{tabular}\\[-0.5ex]
    \begin{tabular}{C{0.8cm}C{0.8cm}C{0.8cm}C{0.5cm}C{0.8cm}}
       & & $+$ &  & $+$
    \end{tabular}\\[-0.5ex]
    \begin{tabular}{C{0cm}C{0.8cm}C{0.8cm}C{0.8cm}C{0.5cm}C{0.8cm}C{0cm}}
       & & & $u_2v_0$ & $\ldots$ & $u_2v_{m-3}$ & $\ldots$
    \end{tabular}\\[-0.6ex]
    \begin{tabular}{C{0.8cm}C{0.8cm}C{0.8cm}C{0.5cm}C{0.8cm}}
       & & & $\ddots$ & $\vdots$
    \end{tabular}\\[-0.6ex]
    \begin{tabular}{C{0cm}C{0.8cm}C{0.8cm}C{0.8cm}C{0.5cm}C{0.8cm}C{0cm}}
       & & & & & $u_{m-1}v_0$ & $\ldots$
    \end{tabular}
  \end{tabular}
  \caption{\footnotesize Visualization of classical multiplication.}
  \label{fig:tiledmult}
  \end{figure}
\end{minipage}
\begin{minipage}{0.49\textwidth}
    \centering
\begin{equation}
\label{eq:clasmul}
u \cdot v = \sum_{k=0}^{m-1} \left( \sum_{\substack{0\leq i,j < m\\i+j=k}}u_i\cdot v_j \right)B^{k}
\end{equation}
\end{minipage}
}

Now, sequentialization is straightforward, but in order to parallelize it, we
make three new observations: All of the tiled multiplications are unique
(i.e. $O(m^2)$ multiplications are necessary), the number of multiplications is
linear in $k$, and the overflows can be added as a seperate step after the
convolutions (e.g. using \textit{badd}).

Since we parallelize at block-level (i.e. with dimensions over the size $m$),
the first observation tells us that each thread must run $O(m)$ multiplications
with no parallelism. However, from the second observation, letting thread
$t_{k\in\{0,..,m-1\}}$ compute digit $w_k$ gives unbalanced amount of sequential
work. Instead, we make a new observation; the work of digits $w_0$ and $w_{m-1}$
is equal to that of $w_1$ and $w_{m-2}$, $w_2$ and $w_{m-3}$, etc. Thus, by
introducing a fixed sequentialization factor of $2$, we can balance the work
amongst threads.

In order to propagate the overflows in a seperate sweep, we must keep track of
them while computing the convolutions. Each thread computes $m+1$
multiplications (when the sequentialization factor is 2). It is a known rule
that; given any two factors, their product fit in their combined
precision. E.g. in decimal system, $9$ (precision 1) times $99$ (precision 2) is
$891$ (precision 3). From this rule, each product can fit in two words. We say
the least significant word is the \textit{low part} and the most significant the
\textit{high part}. Since we add $m+1$ products, each part may overflow at most
$m$ times, s.t. low part overflows carry over to the high part, and high part
overflows carry over to the \textit{carry part}. Hence, we need three words to
keep track of each convolution (assuming B is big enough to hold the carry
part).

Thus, we define the parallel algorithm for classical \textit{mul}tiplication by
\textit{conv}olution given in Figure \ref{fig:mulparalg}, from now on called
\textit{convmul}. This algorithm has work $O(m^2)$ and span $O(m)$, where the
sequential convolution is the dominant factor of the span.

\begin{figure}
\begin{lstlisting}[language=pseudo,escapeinside={(*}{*)},frame=single]
fun conv t = -- Parameter `t` represents current index of the m/2 threads
    k1 = t
    k2 = m - 1 - k1            -- The indices `k1` and `k2` represents the
    l1, l2, h1, h2, c1, c2 = 0 -- upper and lower `k` handled by thread `t`

    for i in (0..k1)        -- The indices `i` and `j` are computed
        j = k1 - i          -- straightforward w.r.t. Equation ((*{\hypersetup{allcolors=ForestGreen}\ref{eq:clasmul}}*))
        l1 += u[i] *(*$_{\mathtt{low}}$*) v[j]
        h1 += u[i] *(*$_{\mathtt{high}}$*) v[j] + overflow(*$_{\mathtt{l1}}$*)
        c1 += overflow(*$_{\mathtt{h1}}$*)

    for i in (0..k2)
        j = k2 - i
        l2 += u[i] *(*$_{\mathtt{low}}$*) v[j]
        h2 += u[i] *(*$_{\mathtt{high}}$*) v[j] + overflow(*$_{\mathtt{l2}}$*)
        c2 += overflow(*$_{\mathtt{h2}}$*)

    return (l1, l2, h1, h2, c1, c2)

(l1, l2, h1, h2, c1, c2) = map conv (0..(m/2)-1)
l = concat l1 (reverse l2)           -- The second half of the
h = shift (concat h1 (reverse h2)) 1 -- convolutions are computed in
c = shift (concat c1 (reverse c2)) 2 -- reverse order due to work balancing
r = ADD l h
w = ADD r c
\end{lstlisting}
  \caption{\footnotesize Pseudocode of work balanced parallel algorithm for classical multiplication by convolution of big integers. \texttt{Input:} $u$ and $v$ of size $m$. \texttt{Output:} $w$ of size $m$. \texttt{Use:} Big integer addition function \texttt{ADD}.}
  \label{fig:mulparalg}
\end{figure}

While we cannot optimize the assymptotics of the algorithm, we can eliminate one
of the additions, increasing the parallelism by eliminating a scan.

The intuition is to compute two lower and upper indices of $k$ in Equation
(\ref{eq:clasmul}) per thread, four indices in total. This allows premature
computation of half the additions before the values leave register memory
(i.e. computed by \texttt{conv}), thus, necessitating only one addition after
the convolutions. This optimization also reduces memory usage.

Figure \ref{fig:muloptmem} contains an illustration of the tiling layout of this
optimization w.r.t. threads. Now, the memory layout is trickier to construct
compared to that of Figure \ref{fig:tiledmult}, requiring threads to alternate
writing to the additive arrays. E.g. with four threads ($m = 16$) the addition
corresponds to memory layout:\\

\begin{center}
  \small
  \begin{tabular}{c}
    \begin{tabular}{|C{0.4cm}|C{0.4cm}|C{0.4cm}|C{0.4cm}|C{0.4cm}|C{0.4cm}|C{0.4cm}|C{0.4cm}|C{0.4cm}|C{0.4cm}|C{0.4cm}|C{0.4cm}|C{0.4cm}|C{0.4cm}|C{0.4cm}|C{0.4cm}|C{0.4cm}|C{0.4cm}|C{0.4cm}|C{0.4cm}|C{0.4cm}|C{0.4cm}|C{0.4cm}|C{0.4cm}|}
      \hline
      \color{Crimson}$t^0_{00}$ & \color{Crimson}$t^0_{01}$ & \color{Crimson}$t^0_{02}$ & \color{Crimson}$t^0_{03}$ & \color{ForestGreen}$t^2_{00}$ & \color{ForestGreen}$t^2_{01}$ & \color{ForestGreen}$t^2_{02}$ & \color{ForestGreen}$t^2_{03}$ & \color{Chocolate}$t^3_{10}$ & \color{Chocolate}$t^3_{11}$ & \color{Chocolate}$t^3_{12}$ & \color{Chocolate}$t^3_{13}$ & \color{RoyalBlue}$t^1_{10}$ & \color{RoyalBlue}$t^1_{11}$ & \color{RoyalBlue}$t^1_{12}$ & \color{RoyalBlue}$t^1_{13}$\\
      \hline
    \end{tabular}\\
    \begin{tabular}{C{0.41cm}C{0.41cm}C{0.41cm}C{0.41cm}C{0.41cm}C{0.41cm}C{0.41cm}C{0.41cm}C{0.41cm}C{0.41cm}C{0.41cm}C{0.41cm}C{0.41cm}C{0.41cm}C{0.41cm}C{0.41cm}C{0.41cm}C{0.41cm}C{0.41cm}C{0.41cm}C{0.41cm}C{0.41cm}C{0.41cm}C{0.41cm}}
      $+$ & $+$ & $+$ & $+$  & $+$ & $+$ & $+$ & $+$  & $+$ & $+$ & $+$ & $+$  & $+$ & $+$ & $+$ & $+$
    \end{tabular}\\
    \begin{tabular}{|C{0.4cm}|C{0.4cm}|C{0.4cm}|C{0.4cm}|C{0.4cm}|C{0.4cm}|C{0.4cm}|C{0.4cm}|C{0.4cm}|C{0.4cm}|C{0.4cm}|C{0.4cm}|C{0.4cm}|C{0.4cm}|C{0.4cm}|C{0.4cm}|C{0.4cm}|C{0.4cm}|C{0.4cm}|C{0.4cm}|C{0.4cm}|C{0.4cm}|C{0.4cm}|C{0.4cm}|}
      \hline
      0 & 0 & \color{RoyalBlue}$t^1_{00}$ & \color{RoyalBlue}$t^1_{01}$ & \color{RoyalBlue}$t^1_{02}$ & \color{RoyalBlue}$t^1_{03}$ & \color{Chocolate}$t^3_{00}$ & \color{Chocolate}$t^3_{01}$ & \color{Chocolate}$t^3_{02}$ & \color{Chocolate}$t^3_{03}$ & \color{ForestGreen}$t^2_{10}$ & \color{ForestGreen}$t^2_{11}$ & \color{ForestGreen}$t^2_{12}$ & \color{ForestGreen}$t^2_{13}$ & \color{Crimson}$t^0_{10}$ & \color{Crimson}$t^0_{11}$\\
      \hline
    \end{tabular}
  \end{tabular}
\end{center}~

The memory layout perfectly partitions into an addition operator with
sequentialization factor of 4, and thus, the optimization efficiently apply at
CUDA block level, yielding both faster addition\footnote{Experimentally, we
  found addition with sequentialization factor of 4 to be the fastest (see
  section \ref{sec:per}).} and elimination of a scan.

\begin{figure}
  {
  \begin{center}
  \small
  \begin{tabular}{C{0.5cm}}
  \Large{\color{Crimson} $t_0$}~)\\\\
    \Large{\color{RoyalBlue} $t_1$}~)
\end{tabular}
  \begin{tabular}{cccccc}
    \color{Crimson}$l_0$ & \color{Crimson}$h_0$ & \color{Crimson}$c_0$ & & & \\
    & \color{Crimson}$l_1$ & \color{Crimson}$h_1$ & \color{Crimson}$c_1$ & & \\
    & & \color{RoyalBlue}$l_2$ & \color{RoyalBlue}$h_2$ & \color{RoyalBlue}$c_2$ & \\
    & & & \color{RoyalBlue}$l_3$ & \color{RoyalBlue}$h_3$ & \color{RoyalBlue}$c_3$
  \end{tabular}
$\xrightarrow{\text{combines to}}$
\begin{tabular}{cccccc}
     & $\xrightarrow{\scriptsize \color{Crimson} \text{carry}}$ & $\xrightarrow{\scriptsize \color{Crimson} \text{carry}}$ &  &  &  \\
    \color{Crimson}$l_0$ & \color{Crimson}$h_0 + l_1$ & \color{Crimson}$c_0 + h_1$ & \color{Crimson}c$_1$ & & \\
    & & \color{RoyalBlue}$l_2$ & \color{RoyalBlue}$h_2+l_3$ & \color{RoyalBlue}$c_2+h_3$ & \color{RoyalBlue} $c_3$\\
  & & & $\color{RoyalBlue}\xrightarrow[\scriptsize \text{carry}]{}$ & $\color{RoyalBlue}\xrightarrow[\scriptsize \text{carry}]{}$ & \\
\end{tabular}
\end{center}
}

~~~~$\vdots$\qquad\qquad\qquad\qquad\qquad $\ddots$ \qquad\qquad\qquad\qquad\qquad\qquad\qquad\qquad\qquad\qquad\qquad\qquad  $\ddots$
  \caption{\footnotesize Illustration of optimized memory layout of \textit{convmul} with sequentialization factor of 4.}
  \label{fig:muloptmem}
\end{figure}


\subsubsection{FFT}
\label{subsubsec:fft}

\subsection{CUDA Implementation}
\label{subsec:mulcud}

In this section, we introduce five CUDA versions, of which two turns out to be
competitive (see Section \ref{sec:per}). The first version (\texttt{V1})
implement the \textit{convmul} algorithm straightforward. The second version
(\texttt{V2}) introduces data type optimization w.r.t. doubling wordsize for
products, and the third version (\texttt{V3}) handles multiple instances per
block to \texttt{V2}. Version four (\texttt{V4}) implements a sequentialization
factor of four according to the optimization described in Section
\ref{subsubsec:classical}, and the fifth (\texttt{V5}) add multiple instances
per block to \texttt{V4}.

The section is divided in three paragraphs, describing the implementations of
convolutions, memory layouts, and multiple instances per block,
respectively. Combined, they describe how we implement the kernels of CUDA
source file \texttt{ker-mul.cu.h}.

\paragraph{Convolutions.}
The implementations use the type \texttt{ubig\_t} of the generic base class,
which is double the base radix (twice the bits of type \texttt{uint\_t}). The
type is used to compute the inner products of the convolution, but it also
allows the optimization of \texttt{V2}. The idea is to postpone the combining of
the low and high parts of the multiplication till after the sequential
loops. Then, the carries can propagate once instead of $i$ times for convolution
$k_i$. The first two convolutions of Listing \ref{cudaconvs} shows the
difference of storing intermediate convolution parts in type \texttt{uint\_t}
(\texttt{V1}) and \texttt{ubig\_t} (\texttt{V2}). The third convolution of
Listing \ref{cudaconvs} shows how to compute and combine two consecutive
convolutions \texttt{(V4)}. It also reveals another upside of this optimization;
we can reuse a digit of $u$, reducing memory usage.

\begin{lstlisting}[language=CPP,escapeinside={(*}{*)},caption={\footnotesize The three convolutions used in the CUDA multiplication kernels of file \texttt{ker-mul.cu.h}.},label={cudaconvs}]
(*{\color{Crimson}$\bullet$ \texttt{Convolution over index k as described in Figure {\hypersetup{allcolors=Crimson}\ref{fig:mulparalg}}}*)
    for (int i=0; i<=k; i++) {
        // compute high and low part of product
        int j = k - i;
        ubig_t uv = ((ubig_t) u[i]) * ((ubig_t) v[j]);
        uint_t l = (uint_t) uv;
        uint_t h = (uint_t) (uv >> Base::bits);
        // update lows, highs, and carries
        ls += l;
        hs += h + (ls < l);
        cs += hs < (h + (ls < l));
    }

(*{\color{Crimson}$\bullet$ \texttt{Convolution over index k optimized for data type ubig\_t}}*)
    ubig_t l = 0;
    ubig_t h = 0;
    for (int i=0; i<=k; i++) {
        // compute high and low part of product
        int j = k - i;
        ubig_t uv = ((ubig_t) u[i]) * ((ubig_t) v[j]);
        l += uv & ((ubig_t) Base::HIGHEST);
        h += uv >> Base::bits;
    }
    // update lows, highs, and carries
    ls = (uint_t) l;
    hs = ((uint_t) h) + ((uint_t) (l >> Base::bits));
    cs = ((uint_t) (h >> Base::bits)) + (hs < ((uint_t) h));

(*{\color{Crimson}$\bullet$ \texttt{Convolution over index k and k+1 optimized for ubig\_t and sequentialization factor of 4}}*)
    ubig_t l1 = 0; ubig_t h1 = 0;
    ubig_t l2 = 0; ubig_t h2 = 0;
    for (int i=0; i<=k; i++) {
        // compute high and low part of product
        int j = k - i;
        ubig_t uv1 = ((ubig_t) u[i]) * ((ubig_t) v[j]);
        ubig_t uv2 = ((ubig_t) u[i]) * ((ubig_t) v[j+1]);
        l1 += uv1 & ((ubig_t) Base::HIGHEST);
        h1 += uv1 >> Base::bits;
        l2 += uv2 & ((ubig_t) Base::HIGHEST);
        h2 += uv2 >> Base::bits;
    }
    // remaining computation where i = k+1
    ubig_t uv = ((ubig_t) u[k+1]) * ((ubig_t) v[0]);
    l2 += uv & ((ubig_t) Base::HIGHEST);
    h2 += uv >> Base::bits;
    // update lows, highs, and carries
    ls1 = (uint_t) l1;
    hs1 = ((uint_t) h1) + ((uint_t) (l1 >> Base::bits));
    cs1 = ((uint_t) (h1 >> Base::bits)) + (hs1 < ((uint_t) h1));
    ls2 = (uint_t) l2;
    hs2 = ((uint_t) h2) + ((uint_t) (l2 >> Base::bits));
    cs2 = ((uint_t) (h2 >> Base::bits)) + (hs2 < ((uint_t) h2));
    // combine the lows, highs and carries
    ls = ls1;
    hsls = hs1 + ls2;
    cshs = cs1 + hs2 + (hsls < hs1);
    cscs = cs2 + (cshs < hs2);
\end{lstlisting}

\paragraph{Memory layout.}
Constructing the memory layout of the original algorithm (Figure
\ref{fig:mulparalg}) is straightforward; the shared memory is partioned in three
equal parts, and each thread write their two convolution parts to each of the
three memory parts. The writes are coalesced by design. E.g. for eight elements
and four threads, we write:
\begin{center}
  \small
  \begin{tabular}{c}
    \begin{tabular}{|C{0.17cm}|C{0.17cm}|C{0.17cm}|C{0.17cm}|C{0.17cm}|C{0.17cm}|C{0.17cm}|C{0.17cm}?C{0.17cm}|C{0.17cm}|C{0.17cm}|C{0.17cm}|C{0.17cm}|C{0.17cm}|C{0.17cm}|C{0.17cm}?C{0.17cm}|C{0.17cm}|C{0.17cm}|C{0.17cm}|C{0.17cm}|C{0.17cm}|C{0.17cm}|C{0.17cm}|}
      \hline
      \color{Crimson}$l^0_{0}$ & \color{RoyalBlue}$l^1_{0}$ & \color{ForestGreen}$l^2_{0}$ & \color{Chocolate}$l^3_{0}$ & \color{Chocolate}$l^3_{1}$ & \color{ForestGreen}$l^2_{1}$ & \color{RoyalBlue}$l^1_{1}$ & \color{Crimson}$l^0_{1}$ & \color{Crimson}$h^0_{0}$ & \color{RoyalBlue}$h^1_{0}$ & \color{ForestGreen}$h^2_{0}$ & \color{Chocolate}$h^3_{0}$ & \color{Chocolate}$h^3_{1}$ & \color{ForestGreen}$h^2_{1}$ & \color{RoyalBlue}$h^1_{1}$ & \color{Crimson}$h^0_{1}$ & \color{Crimson}$c^0_{0}$ & \color{RoyalBlue}$c^1_{0}$ & \color{ForestGreen}$c^2_{0}$ & \color{Chocolate}$c^3_{0}$ & \color{Chocolate}$c^3_{1}$ & \color{ForestGreen}$c^2_{1}$ & \color{RoyalBlue}$c^1_{1}$ & \color{Crimson}$c^0_{1}$ \\
      \hline
    \end{tabular}\\[-0.4ex]
    \begin{tabular}{C{4.45cm}C{4.45cm}C{4.45cm}}
      \upbracefill & \upbracefill & \upbracefill\\[-0.3ex]
      $\mathit{ls}$ & $\mathit{hs}$ & $\mathit{cs}$
    \end{tabular}
  \end{tabular}
\end{center}
Now, reading and adding the three sub-arrays is trivial.

On the other hand, when we optimize the additions for \texttt{V4}, we get a
complicated memory layout (if we are to write/read coalesced). The write is
fairly straighforward, but the read requires more care. E.g. for eight elements and two threads, suppose thread $t^i$ reads four elements to registers $t^i_0$ and four to
$t^i_1$. Then, in order to add the parts correctly, the threads writes and reads in this
pattern:
\begin{center}
  \small
  \begin{tabular}{c}
    \begin{tabular}{C{0.46cm}C{0.46cm}C{0.46cm}C{0.46cm}C{0.46cm}C{0.46cm}C{0.46cm}C{0.46cm}C{0.46cm}C{0.46cm}C{0.46cm}C{0.46cm}C{0.46cm}C{0.46cm}C{0.46cm}C{0.46cm}}
      \color{Crimson}$t^0_{00}$ & \color{Crimson}$t^0_{12}$ & \color{RoyalBlue}$t^1_{00}$ & \color{RoyalBlue}$t^1_{12}$ & \color{Crimson}$t^0_{01}$ & \color{Crimson}$t^0_{13}$ & \color{RoyalBlue}$t^1_{01}$ & \color{RoyalBlue}$t^1_{13}$ & \color{Crimson}$t^0_{02}$ & \color{RoyalBlue}$t^1_{10}$ & \color{RoyalBlue}$t^1_{02}$ & & \color{Crimson}$t^0_{03}$ & \color{RoyalBlue}$t^1_{11}$ & \color{RoyalBlue}$t^1_{03}$ &
    \end{tabular}\\
    \begin{tabular}{|C{0.45cm}|C{0.45cm}|C{0.45cm}|C{0.45cm}?C{0.45cm}|C{0.45cm}|C{0.45cm}|C{0.45cm}?C{0.45cm}|C{0.45cm}|C{0.45cm}|C{0.45cm}?C{0.45cm}|C{0.45cm}|C{0.45cm}|C{0.45cm}|}
      \hline
      \color{Crimson}$l^0_{0}$ & \color{RoyalBlue}$l^1_{0}$ & \color{RoyalBlue}$l^1_{1}$ & \color{Crimson}$l^0_{1}$ & \color{Crimson}$\mathit{lh}^0_{0}$ & \color{RoyalBlue}$\mathit{lh}^1_{0}$ & \color{RoyalBlue}$\mathit{lh}^1_{1}$ & \color{Crimson}$\mathit{lh}^0_{1}$ & \color{Crimson}$\mathit{hc}^0_{0}$ & \color{RoyalBlue}$\mathit{hc}^1_{0}$ & \color{RoyalBlue}$\mathit{hc}^1_{1}$ & \color{Crimson}$\mathit{hc}^0_{1}$ & \color{Crimson}$\mathit{cc}^0_{0}$ & \color{RoyalBlue}$\mathit{cc}^1_{0}$ & \color{RoyalBlue}$\mathit{cc}^1_{1}$ & \color{Crimson}$\mathit{cc}^0_{1}$ \\
      \hline
    \end{tabular}\\[-0.4ex]
    \begin{tabular}{C{3.2cm}C{3.2cm}C{3.2cm}C{3.2cm}}
      \upbracefill & \upbracefill & \upbracefill & \upbracefill\\[-0.3ex]
      $\mathit{ls}$ & $\mathit{hls}$ & $\mathit{chs}$ & $\mathit{ccs}$
    \end{tabular}
  \end{tabular}
\end{center}~

Listing \ref{cudamulmem} in the appendix contains CUDA code for reading and
writing in this pattern.

\paragraph{Multiple instances per block.}


\subsection{Futhark Implementation}
\label{subsec:mulfut}


\subsection{Single Precision Factor}
\label{subsec:mulsin}

Integer multiplication has some special cases; we have multiplication by 0, by 1
and by the radix. Respectively, this is equivalent to 0, identity and a shift
(e.g. $4 \cdot 10^1$ in decimal system is equivalent to $4 \ll 1$ and
$4\cdot 10^2$ to $4\ll 2$ and so forth). Multiplying of big integers has another
special case; multiplication by a single precision factor.

This case can be computed in parallel as an addition with two extra step: I)
Multiply each digit with by the factor giving two arrays with the low and high
parts, respectively. II) Shift the high part array by 1. III) Add the two arrays
using the addition algorithm. Figure \ref{fig:muld} contains the pseudo-code and
illustration of this algorithm.

\begin{figure}
  \centering
  \begin{minipage}{0.45\textwidth}
    \small
    \texttt{Input:} $u$ and of size $m$ base $B$ and digit $d$\\
    \texttt{Output:} $w$ of size $m$ in base $B$\\
    \texttt{Use:} Function \texttt{ADD} for adding big ints
\begin{lstlisting}[language=pseudo,frame=,escapeinside={(*}{*)}]
l = map (*(*$_{\mathtt{low}}$*) d) u
h = map (*(*$_{\mathtt{high}}$*) d) u
h = shift h 1
w = ADD l h
\end{lstlisting}
  \end{minipage}
  \begin{minipage}{0.45\textwidth}
    \centering
    \footnotesize
    \begin{tabular}{c}
      \begin{tabular}{|C{0.7cm}|C{0.7cm}|C{0.7cm}|C{0.7cm}|C{0.7cm}|}
        \hline
        $u_0$ & $u_1$ & $u_2$ & $\cdots$ & $u_{m-1}$\\ 
        \hline
      \end{tabular}\\[-0.3ex]
      \begin{tabular}{C{0.7cm}C{0.7cm}C{0.7cm}C{0.7cm}C{0.7cm}}
        $\cdot$ & $\cdot$ & $\cdot$ & & $\cdot$\\ 
      \end{tabular}\\[-0.7ex]
      \begin{tabular}{C{0.7cm}C{0.7cm}C{0.7cm}C{0.7cm}C{0.7cm}}
        $d$ & $d$ &  $d$ & $\cdots$ &$d$ \\
      \end{tabular}\\[-0.5ex]
      \begin{tabular}{C{0.7cm}C{0.7cm}C{0.8cm}C{0.7cm}C{0.7cm}}
        $=$ & $=$ & $=$ &  & $=$  
      \end{tabular}\\
      \begin{tabular}{|C{0.7cm}|C{0.7cm}|C{0.7cm}|C{0.7cm}|C{0.7cm}|}
        \hline
        $h_{0}$ & $h_1$ & $h_2$ & $\cdots$ & $h_{m-1}$\\
        \hline
      \end{tabular}\\
      \begin{tabular}{C{0.15cm}C{0.15cm}C{0.15cm}C{0.15cm}C{0.15cm}C{0.15cm}C{0.15cm}}
        \diagonalarrowdown{}$_+$ & & \diagonalarrowdown{}$_+$ &  & \diagonalarrowdown{}$_+$  &  & \diagonalarrowdown{}$_+$ \\
      \end{tabular}\\
      \begin{tabular}{|C{0.7cm}|C{0.7cm}|C{0.7cm}|C{0.7cm}|C{0.7cm}|}
        \hline
        $l_{0}$ & $l_1$ & $l_2$ & $\cdots$ & $l_{m-1}$\\
        \hline
      \end{tabular}\\
      \begin{tabular}{C{0.7cm}C{0.7cm}C{0.8cm}C{0.7cm}C{0.7cm}}
        $=$ & $=$ & $=$ &  & $=$  
      \end{tabular}\\
      \begin{tabular}{|C{0.7cm}|C{0.7cm}|C{0.7cm}|C{0.7cm}|C{0.7cm}|}
        \hline
        $w_{0}$ & $w_1$ & $w_2$ & $\cdots$ & $w_{m-1}$\\
        \hline
      \end{tabular}
    \end{tabular}
  \end{minipage}
  \caption{\footnotesize Pseudo-code and illustration of algorithm for parallel multiplication by single precision.}
  \label{fig:muld}
\end{figure}

We could integrate this into the multiple precision multiplication
implementation, by simply checking whether we are in the special case (or any of
the others for that matter). However there are two problems with this: First, we
expect multiple precision to be more dominant than single precision for a
multiple precision arithmetic library, and so it is not deemed worth the extra
computation. Second, this introduces branching into the multiplication, which is
especially troublesome for working with multiple instances per block (as
diverging branches within a block results in unbalanced work among the threads).

Instead, we write a function for \textit{mul}tiplying a big integer by a
\textit{d}igit, called \textit{muld}. Since it is essentially an addition, we
will not benchmark it or discuss it further.

%%% Local Variables:
%%% mode: LaTeX
%%% TeX-master: "../main"
%%% End:
