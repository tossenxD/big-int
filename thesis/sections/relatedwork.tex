\section{Related Work}
\label{sec:rel}

Two works are particularly related to this. The first is ``GPU Implementations
for Midsize Integer Addition and Multiplication'' by Oancea and Watt
\cite{oancea2024gpu}. It has been developed concurrently with this thesis, with
some of their CUDA \cpp\ setup files as starting point of our code
basis.\footnote{\url{https://github.com/coancea/midint-arithmetic}} It shares a
similar approach to addition and classical multiplication as this
thesis. However, where this thesis focus on division, their work focus on FTT
multiplication. They found that FFT multiplication is faster than classical for
big integers of size greater than $2^{15}$ bits. Their approach to FFT involves
finding a finite prime-field that allows the Discrete Fourier Transform (DFT) to
stay in the domain of integers, while simultaneously use base sizes that almost
maps to machine word sizes one-to-one.

The other particularly related work is the state of the art CUDA library called
``Cooperative Groups Big Numbers'' (CGBN), published by NVlabs \cite{CGBN}. CGBN
is aimed at integers in the range of $2^5$ bits to $2^{15}$ bits (i.e.\ small-
to medium-sized integers), where each integer is processed in a cooperative
group of either 4, 8, 16, or 32 threads. Cooperative groups are collections of
threads assigned with special intra-communication properties
\cite{cudaguide}. As evident by the group sizes, CGBN optimize the arithmetics
for warp-level processing.

In comparison to the approach of this thesis (i.e.\ block-level processing of
arithmetics), their approach minimizes latency by storing intermediate results
in local memory -- rather than shared memory -- and it advocates the usage of fast
warp-level instructions. In turn, this allows consecutive arithmetics to fuse
seamlessly, and to be very fast for smaller-sized integers. However, their
approach impose constraints such as being hardware-dependent (i.e.\ the fast
warp-level instructions are specific to the proprietary CUDA platform), offer no
scalability to integers above $2^{15}$ bits (e.g.\ large-sized integers will
exhaust local memory and registers), and requiring the size of integers to be
evenly divisible by 32. \bigskip

Other related works includes a classical and FFT multiplication by Dieguez et
al. \cite{doi:10.1177/10943420221077964}, where the classical multiplication
takes a divide-and-conquer approach s.t.\ convolutions are tiled over CUDA
blocks. This has the benefit of increasing the amount of parallelism within a
block, but at the cost of blocks having to integrate partial convolution results
and carries using atomic operations. To propagate the carries, they use the
hierarchical carry look-ahead scheme that Emmart and Weems use for big integer
addition in \cite{Emmart2010HighPI}. This addition scheme is structured around
propagating the carries at block-level in a bottom-top-bottom fashion: First the
digits are added. Then carries are propagated in threads, then warps, then
block, afterwards to be distributed back in warps, followed by threads. The
blocks can then overflow to the following chunk of digits, which would then
repeat the process.

A more distant related work is Cuda Multiple Precision Arithmetic Library
(CAMPARY) by Joldes et al. \cite{Joldes2016CAMPARYCM}. It aims at small-sized
integers up to a few hundred bits of precision, and use unevaluated sums of
floating-points numbers to represent the integers internally. Hence, it relies
on floating-point arithmetics, rather than exact arithmetics. The idea behind
the number representation is to use algorithms computing the exact error of a
floating-point, and then store the rounded floating-point and the exact error in
two different floating-points. This decomposes their big integer arithmetics to
a series of hardware-supported floating-point arithmetic operations and error
checks. They also support division, based on a similar Newton-Raphson approach
as the division algorithm we use \cite{watt2023efficient}. The algorithms
regarding errors are computationally demanding and the limiting factor for
integer sizes.

%%% Local Variables:
%%% mode: LaTeX
%%% TeX-master: "../main"
%%% End:
