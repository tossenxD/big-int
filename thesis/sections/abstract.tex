\thispagestyle{empty}
\newgeometry{left=5cm,right=5cm,top=5cm}
{\centering
  \section*{Abstract}
}
\label{sec:abstract}
\addcontentsline{toc}{section}{Abstract}
\hfill\\\\

Exact big integer arithmetics is a fundamental component of numerous scientific
fields, and therefore, precondition efficiency. One way to increase efficiency
is by acceleration on GPGPU, calling for parallel arithmetic algorithms. This
thesis examines parallel algorithms for addition, multiplication, and division,
with the premise of fitting in a CUDA block, and consequently, suited for
medium-sized big integers. The algorithms are implemented in the high-level
languages \cpp\ and Futhark. The addition algorithm boils down to a prefix sum,
which runs efficiently in both implementations. The multiplication algorithm is
the classical quadratic method, parallelized by tiling the convolutions to
reduce the sequential work per thread and optimize for locality of
reference. The \cpp\ implementation exhibits good performance, while the Futhark
implementation leaves room for improvement. The division algorithm is based on
finding multiplicative inverses without leaving the domain of big integers. To
do so, a variety of big integer operators and routines are defined, including
shifts, comparisons, and signed subtraction using the prefix sum approach of
addition. The algorithm parameterizes over the involved methods for big integer
arithmetics, and its efficiency directly mirrors the given multiplication
method. In addition to conveying the algorithm, as well as adapting it to big
integers, supplementary implementations have been produced. This includes a
validating and inefficient sequential implementation in C, and a partially
validating and semi-efficient parallel implementation in Futhark.

\restoregeometry

%%% Local Variables:
%%% mode: LaTeX
%%% TeX-master: "../main"
%%% End:
