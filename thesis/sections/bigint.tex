\section{Representation of Big Integers}
\label{sec:big}
When working with big integers, one must make a choice between arbitrary
precision and exact precision integers. Arbitrary precision, like the name
suggests, means that the integers are not bounded by software. This is, for
example, implemented in Haskell \cite{marlow2010haskell}, and provides a great
deal of abstraction, as there can occur no under/overflows or wrap-arounds. In
other words, it allows the programmer to work without worrying about the
sufficiency of the underlying data structure. {\color{red} TODO also mention
  gmp}

However, the cost of this abstraction is inefficiency, since it requires dynamic
memory allocation to handle overflows. This is especially troublesome given a
GPU architecture, as discussed in section \ref{sec:pre}.

Exact precision integers are bounded by a specified size, making them more
suitable for a GPU architecture. The programmer is still allowed to specify
integers of an arbitrary size, and the size of integers can still be changed (by
means of allocating and copying to a new big integer), but the exact size is
always known and remains the same after arithmetic operations. Thus, only exact
precision is consideren in this thesis.

Equation (\ref{eq:rep}) denotes a big integer $u$ in base $B \in \mathbb{N}$ with
$m$ digits $u_{i\in \{0,1,\dots,m-1\}}\in\{0,1,\dots,B-1\}$. E.g. the decimal number
$42$ is equivalent to $2\cdot 10^0+4\cdot 10^1$. {\color{red} TODO add references to
  show that this notation is common}
\begin{equation}
\label{eq:rep}
u = \sum_{i=0}^{m-1}u_i\cdot B^{i}
\end{equation}

The chosen big integer representation (equation (\ref{eq:rep})) translates
naturally on a computer to an array of integer words. Say we specify a big
integer to be an array of 64-bit integers (i.e. $B = 2^{64}$) and precision of
four digits (i.e. array length $m = 4$), then the big integer is equivalent to a
256-bit integer.

Furthermore, we only consider unsigned integers as base, since signed integers
waste space (e.g. there are $m$ ways to write $-1$). The best representation for
signed big integers is by sign-magnitude {\color{red} TODO elaborate}. Thus, we
mostly treat big integers as unsigned, but can easily extend them if we need to.


%%% Local Variables:
%%% mode: LaTeX
%%% TeX-master: "../main"
%%% End:
