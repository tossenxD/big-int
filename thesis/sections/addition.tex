\section{Addition}
\label{sec:add}

Addition is the simplest of the basic arithmetic operators. It is a cornerstone
of arithmetics that can be used to define all other operations - an essential
part of a big integer library.

Former work has been carried out by students as part of a parallel course
project, which will serve as a stepping stone for the algorithm and
implementation of our addition \cite{DPP-project}.

This section is as follows: In subsection \ref{subsec:addalg} we present an
algorithm to compute big integer addition in parallel. In subsection
\ref{subsec:addcud} we discuss how to efficiently implement the algorithm on a
GPU in CUDA, and in subsection \ref{subsec:addfut} how to translate this
implementation to Futhark.

\subsection{Algorithm}
\label{subsec:addalg}

The additions of two big integers is the sum of the addition of their digits:
\begin{equation}
  \label{eq:add}
  u + v = \sum_{i=0}^{m-1}u_i\cdot B^{i} + \sum_{i=0}^{m-1}v_i\cdot B^{i} = \sum_{i=0}^{m-1}(u_i+v_i)\cdot B^{i}
\end{equation}

The $m$ additions of digits translates trivially to a \fun{map} on a
GPU. However, each addition may overflow the base size, which results in a carry
being added to the following digit. In turn, this digit may now overflow, and so
we need to add yet another carry, and so on. The carries will propagate at most
$m$ times, once for each digit, and hence, each digit may need to know whether
digits before it has overflown. This can be efficiently computed as a prefix sum
using a \fun{scan} operator, specifically an exclusive \fun{scan} since the
first digit has no prior information and we do not care if the last digit
overflows. Thus, big integer addition on a GPU is effectively a
\fun{map}-\fun{scan} composition with the following form:

\begin{figure}
  \centering
  \begin{minipage}{0.45\textwidth}
    \small
    \texttt{Input:} $u$ and $v$ of size $m$ base $b$\\
    \texttt{Output:} $w$ of size $m$ in base $b$
\begin{lstlisting}[language=pseudo,frame=]
c = 0
for i in (0..m-1)
    w[i] = u[i] + v[i] + c
    c = if overflow then 1 else 0
\end{lstlisting}
  \end{minipage}
  \begin{minipage}{0.45\textwidth}
    \centering
    \footnotesize
    \begin{tabular}{c}
      \begin{tabular}{|C{0.7cm}|C{0.7cm}|C{0.7cm}|C{0.7cm}|C{0.7cm}|}
        \hline
        $u_0$ & $u_1$ & $u_2$ & $\cdots$ & $u_{m-1}$\\ 
        \hline
      \end{tabular}\\
      \begin{tabular}{C{0.7cm}C{0.7cm}C{0.7cm}C{0.7cm}C{0.7cm}}
        $+$ & $+$ & $+$ & & $+$\\ 
      \end{tabular}\\
      \begin{tabular}{|C{0.7cm}|C{0.7cm}|C{0.7cm}|C{0.7cm}|C{0.7cm}|}
        \hline
        $v_0$ & $v_1$ & $v_2$ & $\cdots$ & $v_{m-1}$\\
        \hline
      \end{tabular}\\
      \begin{tabular}{C{0.7cm}C{0.7cm}C{0.7cm}C{0.7cm}C{0.7cm}}
        $+$ & $+$ & $+$ &  & $+$\\
      \end{tabular}\\
      \begin{tabular}{C{0.7cm}C{0.7cm}C{0.7cm}C{0.7cm}C{0.7cm}}
        $0$ & $c_1$ &  $c_2$ & $\cdots$ &$c_{m-1}$ \\
      \end{tabular}\\[-0.8ex]
      \begin{tabular}{C{0.15cm}C{0.15cm}C{0.15cm}C{0.15cm}C{0.15cm}C{0.15cm}C{0.15cm}}
       \diagonalarrow{} & & \diagonalarrow{} &  & \diagonalarrow{} &  & \diagonalarrow{}\\
      \end{tabular}\\[-2ex]
      \begin{tabular}{C{0.7cm}C{0.7cm}C{0.8cm}C{0.7cm}C{0.7cm}}
        $=$ & $=$ & $=$ &  & $=$  
      \end{tabular}\\
      \begin{tabular}{|C{0.7cm}|C{0.7cm}|C{0.7cm}|C{0.7cm}|C{0.7cm}|}
        \hline
        $w_{0}$ & $w_1$ & $w_2$ & $\cdots$ & $w_{m-1}$\\
        \hline
      \end{tabular}
    \end{tabular}
  \end{minipage}
  \caption{Pseudo-code and illustration of sequential addition algorithm.}
  \label{fig:addseq}
\end{figure}


y

\begin{figure}
  \centering
  \begin{minipage}{0.47\textwidth}
    \footnotesize
    \texttt{Input:} $u$ and $v$ of size $m$ base $b$\\
    \texttt{Output:} $w$ of size $m$ in base $b$\\
    \texttt{Uses:} $f$ to extract the carry of augmented array
\begin{lstlisting}[language=pseudo,frame=,escapeinside={(*}{*)},name=paradd,backgroundcolor=\color{lightgray},]
(r, o) = map2 (*$\oplus$*) u v
\end{lstlisting}
\vspace{-\baselineskip}
\begin{lstlisting}[language=pseudo,frame=,escapeinside={(*}{*)},name=paradd,backgroundcolor=\color{Beige}]
c = scan_exc (*$\otimes$*) e o
\end{lstlisting}
\vspace{-\baselineskip}
\begin{lstlisting}[language=pseudo,frame=,escapeinside={(*}{*)},name=paradd,backgroundcolor=\color{LightSteelBlue}]
w = map2 ((*$\lambda$*) x y (*$\rightarrow$*) x + f y) r c
\end{lstlisting}
\end{minipage}
  \noindent\fcolorbox{white}{lightgray}{%
  \begin{minipage}{0.47\textwidth}
    \centering
    \footnotesize
    \begin{tabular}{c}
      \begin{tabular}{|C{0.7cm}|C{0.7cm}|C{0.7cm}|C{0.7cm}|C{0.7cm}|}
        \hline
        $u_0$ & $u_1$ & $u_2$ & $\cdots$ & $u_{m-1}$\\ 
        \hline
      \end{tabular}\\
      \begin{tabular}{C{0.7cm}C{0.7cm}C{0.7cm}C{0.7cm}C{0.7cm}}
        $\oplus$ & $\oplus$ & $\oplus$ & & $\oplus$\\ 
      \end{tabular}\\
      \begin{tabular}{|C{0.7cm}|C{0.7cm}|C{0.7cm}|C{0.7cm}|C{0.7cm}|}
        \hline
        $v_0$ & $v_1$ & $v_2$ & $\cdots$ & $v_{m-1}$\\
        \hline
      \end{tabular}\\
      \begin{tabular}{C{0.7cm}C{0.7cm}C{0.8cm}C{0.7cm}C{0.7cm}}
        $=$ & $=$ & $=$ &  & $=$  
      \end{tabular}\\
      \begin{tabular}{|C{0.7cm}|C{0.7cm}|C{0.7cm}|C{0.7cm}|C{0.7cm}|}
        \hline
        $r_{0}$ & $r_1$ & $r_2$ & $\cdots$ & $r_{m-1}$\\
        \hline
      \end{tabular}\\
      \begin{tabular}{C{0.15cm}C{0.15cm}C{0.15cm}C{0.15cm}C{0.15cm}C{0.15cm}C{0.15cm}}
        & &  &  &  &  & \\
      \end{tabular}\\[-2ex]
      \begin{tabular}{|C{0.7cm}|C{0.7cm}|C{0.7cm}|C{0.7cm}|C{0.7cm}|}
        \hline
        $o_{0}$ & $o_1$ & $o_2$ & $\cdots$ & $o_{m-1}$\\
        \hline
      \end{tabular}\\
    \end{tabular}
  \end{minipage}
  }\\
  \noindent\fcolorbox{white}{Beige}{%
    \begin{minipage}{0.4525\textwidth}
    \centering
    \scriptsize
    \begin{tabular}{c}
      \begin{tabular}{|C{0.7cm}|C{0.7cm}|C{0.7cm}|C{0.7cm}|C{0.7cm}|}
        \hline
        $o_{0}$ & $o_1$ & $o_2$ & $\cdots$ & $o_{m-1}$\\
        \hline
      \end{tabular}\\
      \begin{tabular}{C{0.15cm}C{0.15cm}C{0.15cm}C{0.15cm}C{0.15cm}C{0.15cm}C{0.15cm}C{0.15cm}C{0.15cm}C{0.15cm}C{0.15cm}}
        \texttt{e}$\shortrightarrow$ & $\otimes$ & $\shortrightarrow$ & $\otimes$ & $\shortrightarrow$ & $\otimes$ & $\shortrightarrow$ & $\otimes$ & $\shortrightarrow$ & $\otimes$ & \\
      \end{tabular}\\[-1.2ex]
      \begin{tabular}{C{0.15cm}C{0.15cm}C{0.15cm}C{0.15cm}C{0.15cm}C{0.15cm}C{0.15cm}C{0.15cm}C{0.15cm}C{0.15cm}C{0.15cm}}
       \diagonalarrowdown{} & & \diagonalarrowdown{} & & \diagonalarrowdown{} &  & \diagonalarrowdown{} &  & \diagonalarrowdown{} & &\\
      \end{tabular}\\[-0.3ex]
      \begin{tabular}{|C{0.7cm}|C{0.7cm}|C{0.7cm}|C{0.7cm}|C{0.7cm}|}
        \hline
        $c_{0}$ & $c_1$ & $c_2$ & $\cdots$ & $c_{m-1}$\\
        \hline
      \end{tabular}
    \end{tabular}
  \end{minipage}
}
\noindent\fcolorbox{white}{LightSteelBlue}{%
  \begin{minipage}{0.47\textwidth}
    \centering
    \footnotesize
    \begin{tabular}{c}
      \begin{tabular}{|C{0.7cm}|C{0.7cm}|C{0.7cm}|C{0.7cm}|C{0.7cm}|}
        \hline
        $r_0$ & $r_1$ & $r_2$ & $\cdots$ & $r_{m-1}$\\ 
        \hline
      \end{tabular}\\
      \begin{tabular}{C{0.7cm}C{0.7cm}C{0.7cm}C{0.7cm}C{0.7cm}}
        $\lambda$ & $\lambda$ & $\lambda$ & & $\lambda$\\ 
      \end{tabular}\\
      \begin{tabular}{|C{0.7cm}|C{0.7cm}|C{0.7cm}|C{0.7cm}|C{0.7cm}|}
        \hline
        $c_0$ & $c_1$ & $c_2$ & $\cdots$ & $c_{m-1}$\\
        \hline
      \end{tabular}\\
      \begin{tabular}{C{0.7cm}C{0.7cm}C{0.8cm}C{0.7cm}C{0.7cm}}
        $=$ & $=$ & $=$ &  & $=$  
      \end{tabular}\\
      \begin{tabular}{|C{0.7cm}|C{0.7cm}|C{0.7cm}|C{0.7cm}|C{0.7cm}|}
        \hline
        $w_{0}$ & $w_1$ & $w_2$ & $\cdots$ & $w_{m-1}$\\
        \hline
      \end{tabular}\\
    \end{tabular}
  \end{minipage}
  }
  \caption{Pseudo-code and illustration of parallel addition algorithm.}
  \label{fig:addpar}
\end{figure}


\begin{equation}
\label{eq:addgpu}
u + v \equiv \fun{map} + \left( \fun{map} + u~v\right) \left(\left({\fun{scan\_exc}~e \otimes \phantom{}} \circ~\fun{map} \oplus \phantom{}\right)~u~v\right)
\end{equation}
The task at hand is to figure out what $\oplus$, $\otimes$ and $e$ is. This is exactly what
Olesen, Topalovic and Restelli-Nielsen did in a course-project on this exact
topic \cite{DPP-PROJECT}. They found that $\oplus$ must compute whether the addition
results in an overflow or the maximum representable integer. Say we have digits
$x$ and $y$ represented as unsigned integers in base $B$ with wrapping on
overflow, then we get:
\begin{equation}
\label{eq:oplus}
x \oplus y \coloneq \tup{x + y < x}{x + y = B-1}
\end{equation}

The operator $\otimes$ then use the inner tuple to judge whether a digit overflows by
I) if it already overflowed, or II) if it is the maximum representable integer
and the digit just before it overflowed. Say we have some pair of booleans
$x = \tup{\mathit{ov_x}}{\mathit{mx_{x}}}$ and some pair of booleans
$y = \tup{\mathit{ov_y}}{\mathit{mx_{y}}}$, then we get:
\begin{equation}
  \label{eq:otimes}
  x \otimes y \coloneq \tup{\mathit{ov_x} \land \mathit{mx_y} \lor \mathit{ov_y}}{\mathit{mx_x} \land \mathit{mx_y}}
\end{equation}

They found that the neutral element, $e$, of $\otimes$ is:
\begin{equation}
  \label{eq:otimesne}
  e \coloneq \tup{\mathbf{F}}{\mathbf{T}}
\end{equation}

While this approach is intuitive and straightforward to implement in Futhark, a
Cuda implementation highlights an optimization vector by the underlying data
types: Hardware does not support boolean values - they are syntactic sugar for
zero-and-nonzero integer words. Since a low-level implementation will have to
use integers, we might as well use bitwise operations over logical ones, as
these are faster.

Furthermore, instead of storing pairs of integer words with only one bit used to
indicate boolean values, we can store one word with two bits used to each
indicate a boolean value. Not only does this halve the memory usage of the
prefix sum, but it also results in each workthread only having to fetch and
write one word instead of two, which should result in better memory utilization.

The formalization of these optimizations, where the least and second least
significant bit indicates integer overflow and maximum representable integer,
respectively, is:
\begin{align}
  \label{eq:oplusopt}
  x \oplus y &\coloneq (x + y < x)~|~((x + y = B-1) \ll 1) \\
  \label{eq:otimesopt}
  x \otimes y &\coloneq (((x~\&~(y \gg 1))~|~y)~\&~1)~|~(x~\&~y~\&~2)\\
  \label{eq:otimesneopt}
  e &\coloneq 2
\end{align}

Equation (\ref{eq:oplusopt}) is a straightforward conversion of equation
(\ref{eq:oplus}), with the tuple being replaced by the bitwise
\fun{or}-operator, and the second result of the second clause shifted one bit to
the left.

Equation (\ref{eq:otimesopt}) is a conversion of equation (\ref{eq:otimes}),
where I) the tuple is replaced by \fun{or}, II) the first clause is checked in
the least significant bit and zeroes out the second bit with ``$\&~1$'', and
III) the second clause is checked in the second least significant bit and the
first bit is zeroed out with ``$\&~2$''. Associtivity naturally still holds for
$\otimes$.

\begin{proof}
  We want to show $x \otimes (y \otimes z) = (x \otimes y) \otimes z$. From left-to-right we have:
\begin{align}
  (x \otimes y) \otimes z &= (~\underbrace{(((x~\&~(y \gg 1))~|~y)~\&~1)~|~(x~\&~y~\&~2)}_{\alpha}~) \otimes z\\
              &= (((\alpha~\&~(z \gg 1))~|~z)~\&~1)~|~(\alpha~\&~z~\&~2)\\
              \label{eq:otimesproofltr}
              &= (((\alpha~\&~1)~\&~(z \gg 1))~|~(z~\&~1))~|~((\alpha~\&~2)~\&~z)
\end{align}
From right-to-left we have:
\begin{align}
  x \otimes (y \otimes z) &= x \otimes (~\underbrace{((y~\&~(z \gg 1))~|~z)~\&~1)~|~(y~\&~z~\&~2)}_{\beta}~)\\
  &= (((x~\&~(\beta \gg 1))~|~\beta)~\&~1)~|~(x~\&~\beta~\&~2)\\
  \label{eq:otimesproofrtl}
  &= ((x~\&~(\beta \gg 1)~\&~1)~|~(\beta~\&~1))~|~(x~\&~(\beta~\&~2))
\end{align}
We must now show that equation (\ref{eq:otimesproofltr}) is equal to
(\ref{eq:otimesproofrtl}). Consider the second clause. Since $(1~\&~2)$ is $0$,
so is the first clause of both $(\alpha~\&~2)$ and $(\beta~\&~2)$. Futhermore, we have
that $2~\&~2$ is the same as $2$. Thus, we have:
\begin{align}
  (\alpha~\&~2)~\&~z &= (x~\&~y~\&~2)~\&~z\\
                &= x~\&~(y~\&~z~\&~2)\\
                &= x~\&~(\beta~\&~2)
\end{align}
Consider the first clause. Again, $(2~\&~1)$ is $0$, so the second clause of $(\alpha~\&~1)$ is $0$:
\begin{align}
  ((\alpha~\&~1)~\&~&(z \gg 1))~|~(z~\&~1) \\
               &= ((((x~\&~(y \gg 1))~|~y)~\&~1)~\&~(z \gg 1))~|~(z~\&~1) \\
               &= (((x~\&~(y \gg 1)~\&~1)~|~(y~\&~1))~\&~(z \gg 1))~|~(z~\&~1) \\
               &= ((x~\&~(y \gg 1)~\&~(z \gg 1)~\&~1)~|~(y~\&~(z \gg 1)~\&~1))~|~(z~\&~1)
\end{align}
We have $((y \gg 1)~\&~(z \gg 1)) = (y~\&~z \gg 1) = (y~\&~z~\&~2 \gg 1)$ and get:
\begin{align}
  &= (x~\&~((y~\&~z~\&~2) \gg 1)~\&~1)~|~(y~\&~(z \gg 1)~\&~1)~|~(z~\&~1))\\
  \phantom{((\alpha~\&~1)\&}&= (x~\&~((y~\&~z~\&~2) \gg 1)~\&~1)~|~((y~\&~(z \gg 1))~|~z)~\&~1)\\
  &= (x~\&~(\beta \gg 1)~\&~1)~|~(\beta~\&~1)
\end{align}
Thus, equations (\ref{eq:otimesproofltr}) and (\ref{eq:otimesproofrtl}) are equal, and hence, $x \otimes (y \otimes z) = (x \otimes y) \otimes z$.
\end{proof}




Likewise, the optimized $e$ (equation (\ref{eq:otimesneopt})) is simply the
corresponding bits of equation (\ref{eq:otimesne}) set to 1 and 0, and so $e$
remains a left-associative neutral element of $\otimes$.

\begin{proof} By exhaustive evaluation we have:
\begin{align}
  \label{eq:otimesneproof}
  2 \otimes 0 &= (((2~\&~(0 \gg 1))~|~0)~\&~1)~|~(2~\&~0~\&~2) = 0 \\
  2 \otimes 1 &= (((2~\&~(1 \gg 1))~|~1)~\&~1)~|~(2~\&~1~\&~2) = 1 \\
  2 \otimes 2 &= (((2~\&~(2 \gg 1))~|~2)~\&~1)~|~(2~\&~2~\&~2) = 2 \\
  2 \otimes 3 &= (((2~\&~(3 \gg 1))~|~3)~\&~1)~|~(2~\&~3~\&~2) = 3
\end{align}
\end{proof}

Since we needed extra information (the second least significant bit) in order
for $\otimes$ to be associative, we need to discard it after the \fun{scan}. Thus, the
addition becomes:
\begin{equation}
\label{eq:addgpu1}
u + v \equiv \fun{map} + \left( \fun{map} + u~v\right)~\left(\left(\fun{map}~(\&~1) ~\circ ~{\fun{scan\_exc}~e~\otimes}~ \circ ~ \fun{map} \oplus \phantom{}\right)~u~v\right)
\end{equation}

Both $\oplus$ and $\otimes$ are $O(1)$, and so the \fun{map}s exhibit total work
$O(m)$ and total span $O(1)$, and the \fun{scan} work $O(m)$ and span
$O(\log m)$. Thus, the work of a big integer addition is $O(m)$ and the span is
$O(\log m)$.


\subsection{CUDA Implementation}
\label{subsec:addcud}

{\color{red} TODO write about further optimizitations, e.g. upping the sequentialization factor}

\subsection{Futhark Implementation}
\label{subsec:addfut}

%%% Local Variables:
%%% mode: LaTeX
%%% TeX-master: "../main"
%%% End:
