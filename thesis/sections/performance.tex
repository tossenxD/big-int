\section{Performance}
\label{sec:per}

{\red [TODO; rewrite]}

The main benefit of data parallel computing is superior scaling and performance
gain on big data. As such, the big integer library written in this project must
be efficient and perform well on medium-sized big integers, but also scale well
across the size of the big integers, the number of big integer instances and the
number of operations.

In order to determine the efficiency of the Futhark library, we benchmark the
performance and compare it to our CUDA implementation. The assumption is that
the CUDA implementation yields the best performance possible, given the
underlying design, algorithms and optimizations, revealing the performance
overhead and inefficiencies of the Futhark library. In turn, the efficiency of
the underlying basis is just as important, and so we compare it against the
state of the art CUDA library CGBN.

Efficient computation means multiple things; a low power consumption, good
memory usage, good thread utilization, low running time, high bandwidth,
operations per second, etc. In the spirit of Futhark, we are interested in the
hardware-agnostic performance. A good ground is benchmarking the running time,
as the relative running times should be similar across hardware (assuming it
scales linearly on the specifications of the underlying GPUs).

However, running times are not directly comparable unless we keep the data size
fixed. Instead, we define some performance metrics based on the running time,
big integer size, number of instances and more, allowing us to compare versions
and implementations in an objective manner.

Section \ref{subsec:benchset} describes the benchmark setup, section
\ref{subsec:perfmet} introduces the performance metrics, and section
\ref{subsec:benchres} presents and discuss the benchmark results.

\subsection{Benchmark Setup}
\label{subsec:benchset}

{\red [TODO; rewrite]}

In order to understand the performance and scaling thereof, two benchmark
structures are defined for each arithmetic function. The first structure is
simply the functions themselves, which serves as the basis for judging the
performance. The second structure is multiple consecutive calls to the
functions, designed to give insight in the scaling of the functions in
semi-realistic usecases. These structures are generic over the big integer
precision and number of arithmetic instances.

While we want to vary the number of instances and precision, the benchmark
results are only directly comparable if we keep the total amount of work
fixed. E.g. consider a fixed 32-bit digit work size of 128; the benchmarks can
then be run with 1 instance of base 32-bit 128-precision big integers, 1
instance of base 64-bit 64-precision big integers, 2 instances of base 32-bit
64-precision big integers, etc., which all uses the exact same number of
bits. As such, the benchmark results will explicitly reveal which
function-version is most suitable on what input pattern.

The second benchmark structure is defined below, where program 1 and 2 is for
addition and multiplication, respectively. The programs corresponds to
calculating $(10 \cdot u + v)$ and $(u \cdot v)^5$, respectively. {\color{red} TODO
  bench 2 should probably be like described here, but is a bit different in the
  code}

\begin{lstlisting}[language=futhark]
loop w = v for i < 10 do map2 add u w
let s = map2 mul u v in loop w = s for i < 5 do map2 mul s w
\end{lstlisting}

The chosen fixed total arithmetic work of 32-bit digits is $134217728$, and run
with base 32-bit big integer sizes $16, 32, 64, \ldots, 2048$ and half of that for
64-bit base benchmarks. Runtimes are averaged over multiple runs to improve
stability. The benchmarks are run on a NVIDIA GTX 1650 SUPER - a GPU with 1280
CUDA cores, 4GB memory and 192.0 GB/s memory bandwidth \cite{gpuspecs}. Futhark
functions are benchmarked using \texttt{futhark-bench} and can be adjusted in
the directory \texttt{bench}, whereas CUDA benchmarks are both run and adjusted
directly from the mainfile \texttt{main.cu}. Each part contains a
\texttt{Makefile} to replicate the benchmarks.
\subsection{Performance Metrics}
\label{subsec:perfmet}

{\red [TODO; rewrite (mostly okay)]}

Performance metrics are defined on an arithmetic basis, based on their
complexity, inner workings and benchmark setup. Runtimes are measured in
microseconds.

\begin{itemize}[leftmargin=*]
\item \textbf{Addition}\\
  In Section \ref{sec:add} we analyze the complexity of addition to work $O(n)$
  and span $O(\log n)$. It is computational efficient, and only a single
  \fun{scan} function exceeds $O(1)$ span. Hence, we expect addition to be
  \textit{bandwidth} bound rather than \textit{operation} bound.

  For $n$ instances of addition with $m$-sized base $b$ big integers we
  calculate the bandwidth, measured in gigabytes/second, using the following
  formula:
  \begin{equation}
    \label{eq:bandwidth}
    \mathit{bandwidth} = 3 \cdot \dfrac{n \cdot m \cdot \log(b)/8}{1000000000} \cdot \dfrac{1000000}{\mathit{runtime}}
  \end{equation}
  The first fraction represents how many gigabytes of a big integer are
  accessed, where the term $\log(b) / 8$ is the bytesize of a digit, $n\cdot m$ is
  the amount of accessed digits and the divisor converts to gigabytes. The
  second fraction is the runtime converted to seconds. The number 3 is because
  it must at minimum fetch twice and write once to/from global memory,
  regardless of performing one or ten consecutive additions. Thus, we use the
  same formula for both benchmark setups.
  

\item \textbf{Multiplication}\\
  In Section \ref{sec:mul} we analyze the complexity of multiplication to work
  $O(n^2)$ and span $O(n)$. It is much more computational demanding than
  addition, and so we expect to be \textit{operation} bound.

  Thus, we estimate the number of operations per second. In order to better
  compare different bases, we specifically chose the number of unsigned 32-bit
  integer operations per second as the metric.
  \begin{equation}
  \label{eq:u32ops}
  \mathit{Gu32ops} = \dfrac{n \cdot (m \cdot \log(b) / 32)^2}{1000000000} \cdot \dfrac{1000000}{\mathit{runtime}}
  \end{equation}
  The first fraction is the number of 32-bits in the big integer squared,
  multiplied by the number of big integers and converted to giga-operations. The
  second fraction is the runtime converted to seconds. When we compute six
  consecutive multiplications, we execute six times the number of operations,
  and so we multiple the metric by 6.

  {\color{red} TODO I think it is a bit different in the implementation (by a
    constant factor), so either update this or the implementation}
\end{itemize}

The performance metrics are computed directly as part of the benchmarking for
CUDA in \texttt{main.cu}, but \texttt{futhark-bench} reports only the average
runtimes (with confidence intervals). To compute the metrics for Futhark
benchmarks, a small AWK program \texttt{form-tool.awk} is included in the
Futhark directory and integrated in the \texttt{Makefile}'s benchmark call.
\pagebreak
\subsection{Addition Benchmark Results}
\label{subsec:benchres}

\begin{table}
  \centering
  \begin{tabular}{|c|c||c?c|c|c?c|c|c|c|}\hline
    Bits & I{\footnotesize nstances} & CGBN & C-\texttt{V1} & C-\texttt{V2} & C-\texttt{V3}  & F-\texttt{V0} & F-\texttt{V1} & F-\texttt{V2} & F-\texttt{V3}\\\hline\hline
    $2^{18}$ & $2^{14}$ & 62  & --   & 146 & 161 & --   & --   & --   & --   \\\hline
    $2^{17}$ & $2^{15}$ & 67  & --   & 161 & 163 & --   & --   & --   & --   \\\hline
    $2^{16}$ & $2^{16}$ & 19  & 168 & 166 & 166 & 116 & 154 & 158 & 156 \\\hline
    $2^{15}$ & $2^{17}$ & 19  & 168 & 167 & 166 & 145 & 168 & 168 & 146 \\\hline
    $2^{14}$ & $2^{18}$ & 84  & 168 & 168 & 166 & 150 & 168 & 168 & 168 \\\hline
    $2^{13}$ & $2^{19}$ & 164 & 168 & 168 & 164 & 156 & 168 & 168 & 168 \\\hline
    $2^{12}$ & $2^{20}$ & 165 & 168 & 168 & 166 & 156 & 168 & 142 & 168 \\\hline
    $2^{11}$ & $2^{21}$ & 164 & 169 & 163 & 166 & 134 & 165 & 75  & 168 \\\hline
    $2^{10}$ & $2^{22}$ & 157 & 108 & 93  & 166 & 71  & 91  & 38  & 168 \\\hline
    $2^{9}$  & $2^{23}$ & 119 & 55  & 47  & 167 & 36  & 47  & 20  & 168 \\\hline
  \end{tabular}
  \caption{\footnotesize Performance of one addition in base \texttt{u64} measured in GB/s (higher is better, 192 is peak)}
\end{table}

\begin{table}
  \centering
  \begin{tabular}{|c|c||c?c|c|c?c|c|c|c|}\hline
    Bits & I{\footnotesize nstances} & CGBN & C-\texttt{V1} & C-\texttt{V2} & C-\texttt{V3}  & F-\texttt{V0} & F-\texttt{V1} & F-\texttt{V2} & F-\texttt{V3}\\\hline\hline
    $2^{18}$ & $2^{14}$ & 24  & --  & 92  & 83  & --  & --  & --  & --  \\\hline
    $2^{17}$ & $2^{15}$ & 57  & --  & 111 & 94  & --  & --  & --  & --  \\\hline
    $2^{16}$ & $2^{16}$ & 73  & 41 & 123 & 107 & --  & --  & --  & --  \\\hline
    $2^{15}$ & $2^{17}$ & 45  & 48 & 128 & 107 & 15 & 24 & 15 & 13 \\\hline
    $2^{14}$ & $2^{18}$ & 98  & 51 & 112 & 107 & 18 & 27 & 16 & 13 \\\hline
    $2^{13}$ & $2^{19}$ & 162 & 46 & 83  & 108 & 18 & 30 & 14 & 13 \\\hline
    $2^{12}$ & $2^{20}$ & 164 & 35 & 44  & 108 & 16 & 29 & 14 & 13 \\\hline
    $2^{11}$ & $2^{21}$ & 162 & 24 & 23  & 108 & 15 & 25 & 10 & 13 \\\hline
    $2^{10}$ & $2^{22}$ & 153 & 12 & 12  & 108 & 7  & 13 & 5  & 13 \\\hline
    $2^{9}$  & $2^{23}$ & 110 & 6  & 6   & 108 & 4  & 7  & 3  & 13 \\\hline
  \end{tabular}
  \caption{\footnotesize Performance of ten additions in base \texttt{u64} measured in GB/s (higher is better, 192 is peak)}
\end{table}


{\red [Missing]}

\subsection{Multiplication Benchmark Results}
\label{subsec:benchres}

{\red [Missing]}

%%% Local Variables:
%%% mode: LaTeX
%%% TeX-master: "../main"
%%% End:
