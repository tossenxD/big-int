\section*{Appendix}

\subsection*{A}
\label{app:A}

\begin{proof}[Proof of associativity for operator $\otimes$ of Equation (\ref{eq:otimesopt})]\phantom{~}
  
  We want to show:
\begin{equation}
  \label{eq:aconc}
  x \otimes (y \otimes z) = (x \otimes y) \otimes z
\end{equation}
From right-to-left we have:
\begin{align}
  (x \otimes y) \otimes z &= (~\underbrace{(((x~\&~(y \gg 1))~|~y)~\&~1)~|~(x~\&~y~\&~2)}_{\alpha}~) \otimes z\\
              &= (((\alpha~\&~(z \gg 1))~|~z)~\&~1)~|~(\alpha~\&~z~\&~2)\\
              \label{eq:otimesproofltr}
              &= (((\alpha~\&~1)~\&~(z \gg 1))~|~(z~\&~1))~|~((\alpha~\&~2)~\&~z)
\end{align}
From left-to-right we have:
\begin{align}
  x \otimes (y \otimes z) &= x \otimes (~\underbrace{((y~\&~(z \gg 1))~|~z)~\&~1)~|~(y~\&~z~\&~2)}_{\beta}~)\\
  &= (((x~\&~(\beta \gg 1))~|~\beta)~\&~1)~|~(x~\&~\beta~\&~2)\\
  \label{eq:otimesproofrtl}
  &= ((x~\&~(\beta \gg 1)~\&~1)~|~(\beta~\&~1))~|~(x~\&~(\beta~\&~2))
\end{align}
We must now show that Equation (\ref{eq:otimesproofltr}) is equal to
(\ref{eq:otimesproofrtl}). Consider the second clause. Since $(1~\&~2)$ is $0$,
so is the first clause of both $(\alpha~\&~2)$ and $(\beta~\&~2)$. Furthermore, we have
that $2~\&~2$ is the same as $2$. Thus, we have:
\begin{align}
  (\alpha~\&~2)~\&~z &= (x~\&~y~\&~2)~\&~z\\
                &= x~\&~(y~\&~z~\&~2)\\
                &= x~\&~(\beta~\&~2)
\end{align}
Consider the first clause. Again, $(2~\&~1)$ is $0$, so the second clause of $(\alpha~\&~1)$ is $0$:
\begin{align}
  ((\alpha~\&~1)~\&~&(z \gg 1))~|~(z~\&~1) \\
               &= ((((x~\&~(y \gg 1))~|~y)~\&~1)~\&~(z \gg 1))~|~(z~\&~1) \\
               &= (((x~\&~(y \gg 1)~\&~1)~|~(y~\&~1))~\&~(z \gg 1))~|~(z~\&~1) \\
               &= ((x~\&~(y \gg 1)~\&~(z \gg 1)~\&~1)~|~(y~\&~(z \gg 1)~\&~1))~|~(z~\&~1)
\end{align}
We have $((y \gg 1)~\&~(z \gg 1)) = (y~\&~z \gg 1) = (y~\&~z~\&~2 \gg 1)$ and get:
\begin{align}
  &= (x~\&~((y~\&~z~\&~2) \gg 1)~\&~1)~|~(y~\&~(z \gg 1)~\&~1)~|~(z~\&~1))\\
  \phantom{((\alpha~\&~1)\&}&= (x~\&~((y~\&~z~\&~2) \gg 1)~\&~1)~|~((y~\&~(z \gg 1))~|~z)~\&~1)\\
  &= (x~\&~(\beta \gg 1)~\&~1)~|~(\beta~\&~1)
\end{align}
Thus, Equations (\ref{eq:otimesproofltr}) and (\ref{eq:otimesproofrtl}) are equal, and hence, (\ref{eq:aconc}) holds and $\otimes$ is associative.
\end{proof}

\begin{proof}[Proof that (\ref{eq:otimesneopt}) is left-associative neutral element for (\ref{eq:otimesopt})]\phantom{~}

  By exhaustive evaluation we have:
\begin{align}
  \label{eq:otimesneproof}
  2 \otimes 0 &= (((2~\&~(0 \gg 1))~|~0)~\&~1)~|~(2~\&~0~\&~2) = 0 \\
  2 \otimes 1 &= (((2~\&~(1 \gg 1))~|~1)~\&~1)~|~(2~\&~1~\&~2) = 1 \\
  2 \otimes 2 &= (((2~\&~(2 \gg 1))~|~2)~\&~1)~|~(2~\&~2~\&~2) = 2 \\
  2 \otimes 3 &= (((2~\&~(3 \gg 1))~|~3)~\&~1)~|~(2~\&~3~\&~2) = 3
\end{align}
\end{proof}

\subsection*{B}
\label{app:B}

\begin{lstlisting}[language=CPP,caption={\footnotesize CUDA convolution memory
transactions of \textit{convmul} with sequentialization factor of 4 to prepare
calling \textit{badd} on the four pairs of parts in registers, from file
\texttt{ker-mul.cu.h} (slightly edited), with base class \texttt{Base}, big integer size \texttt{m}, and \texttt{ipb} instances per block.},label={cudamulmem},firstnumber=106]{../cuda/ker-mul.cu.h}
cp4Regs2Shm(typename Base::uint_t* lhc0, typename Base::uint_t* lhc1,
            typename Base::uint_t* shmem){
    uint32_t off = threadIdx.x;
    uint32_t str = ipb * m/2;
    shmem[off] = lhc0[0];
    shmem[off+str] = lhc0[1];
    shmem[off+2*str] = lhc0[2];
    shmem[off+3*str] = lhc0[3];

    shmem[str-1-off] = lhc1[0];
    shmem[2*str-1-off] = lhc1[1];
    shmem[3*str-1-off] = lhc1[2];
    shmem[4*str-1-off] = lhc1[3];
}

cpShm24Regs(typename Base::uint_t* shmem, typename Base::uint_t* lhc0,
            typename Base::uint_t* lhc1){
    uint32_t off = threadIdx.x*2;
    uint32_t off_inst = 2*off % m;
    uint32_t str = ipb * m/2;
    lhc0[0] = shmem[off];
    lhc0[1] = shmem[off+str];
    lhc0[2] = shmem[off+2*str];
    lhc0[3] = shmem[off+3*str];

    lhc1[2] = shmem[off+1];
    lhc1[3] = shmem[off+str+1];
    lhc1[0] = (off_inst) ? shmem[off+2*str-1] : 0;
    lhc1[1] = (off_inst) ? shmem[off+3*str-1] : 0;
}
\end{lstlisting}

\bigskip

\subsection*{C}
\label{app:C}

\begin{proof}[Proof of associativity for operator $\odot$ of Equation (\ref{eq:lt})]\phantom{~}

  We want to show:
  \begin{equation}
    \label{eq:ccon}
    \tup{\mathit{l1}}{\mathit{e1}} \odot (\tup{\mathit{l2}}{\mathit{e2}} \odot \tup{\mathit{l3}}{\mathit{e3}}) = (\tup{\mathit{l1}}{\mathit{e1}} \odot \tup{\mathit{l2}}{\mathit{e2}}) \odot \tup{\mathit{l3}}{\mathit{e3}}
\end{equation}
  By definition we get from left-to-right:
\begin{align}
  \tup{\mathit{l1}}{\mathit{e1}} \odot (&\tup{\mathit{l2}}{\mathit{e2}} \odot \tup{\mathit{l3}}{\mathit{e3}})\\
  &= \tup{\mathit{l1}}{\mathit{e1}} \odot \tup{\mathit{l3} \lor (\mathit{e3} \land \mathit{l2})}{\mathit{e2} \land \mathit{e3}}\\
  &= \tup{(\mathit{l3} \lor (\mathit{e3} \land \mathit{l2})) \lor ((\mathit{e2} \land \mathit{e3}) \land \mathit{l1})}{\mathit{e1} \land (\mathit{e2} \land \mathit{e3})}\\
  &= \tup{(\mathit{l3} \lor (\mathit{e3} \land (\mathit{l2} \lor (\mathit{e2} \land \mathit{l1})))}{(\mathit{e1} \land \mathit{e2}) \land \mathit{e3}}\\
  &= \tup{\mathit{l2} \lor (\mathit{e2} \land \mathit{l1})}{\mathit{e1} \land \mathit{e2}} \odot \tup{\mathit{l3}}{\mathit{e3}}\\
  &= (\tup{\mathit{l1}}{\mathit{e1}} \odot \tup{\mathit{l2}}{\mathit{e2}}) \odot \tup{\mathit{l3}}{\mathit{e3}}\label{eq:ccon2}
\end{align}
Equation (\ref{eq:ccon2}) is equal to the right-hand-side of (\ref{eq:ccon}), and thus, operator $\odot$ is associative.
\end{proof}

\bigskip

\begin{proof}[Proof that $\tup{\mathtt{F}}{\mathtt{T}}$ is left-associative neutral element for $\odot$ of (\ref{eq:lt})]\phantom{~}

By exhaustive evaluation we have:
\begin{align}
  \label{eq:otimesneproof}
  \tup{\mathtt{F}}{\mathtt{T}} \odot \tup{\mathtt{F}}{\mathtt{F}} &= \tup{\mathtt{F} \lor (\mathtt{F} \land \mathtt{F})}{\mathtt{F} \land \mathtt{F}} = \tup{\mathtt{F}}{\mathtt{F}} \\
  \tup{\mathtt{F}}{\mathtt{T}} \odot \tup{\mathtt{F}}{\mathtt{T}} &= \tup{\mathtt{F} \lor (\mathtt{T} \land \mathtt{F})}{\mathtt{T} \land \mathtt{T}} = \tup{\mathtt{F}}{\mathtt{T}} \\
  \tup{\mathtt{F}}{\mathtt{T}} \odot \tup{\mathtt{T}}{\mathtt{F}} &= \tup{\mathtt{T} \lor (\mathtt{F} \land \mathtt{F})}{\mathtt{T} \land \mathtt{F}} = \tup{\mathtt{T}}{\mathtt{F}} \\
  \tup{\mathtt{F}}{\mathtt{T}} \odot \tup{\mathtt{T}}{\mathtt{T}} &= \tup{\mathtt{T} \lor (\mathtt{T} \land \mathtt{F})}{\mathtt{T} \land \mathtt{T}} = \tup{\mathtt{T}}{\mathtt{T}}
\end{align}
\end{proof}

%%% Local Variables:
%%% mode: LaTeX
%%% TeX-master: "../main"
%%% End:
